\chapter{无网格法优化的数值积分方案}\label{C}
本附录列出了赫林格-赖斯纳变分原理下的本质边界条件施加方法在数值实现过程中所采用的积分点位置和权重,其中表\ref{Ctwo}、\ref{Cfour}分别为二次梯度和四次梯度情况下的优化数值积分方案,
表中$\xi$、$\eta$和$\gamma$为三角形参数空间坐标,$w$和$w_B$分别为三角形域内积分权重和边界积分权重。
\begin{table}[H]
    \caption{\textbf{无网格法优化的数值积分方案——二次梯度}}\label{Ctwo}
    \centering
    \begin{tabular}{cccccc}
       \toprule
       数值积分点&$\xi$ & $\eta$ & $\gamma$ & $w$ & $w_B$\\
       \midrule
       \begin{minipage}[b]{0.3\columnwidth}
        \centering
        \raisebox{-.5\height}{\includegraphics{figure/E/point.png}}
    \end{minipage}&
       $\frac{2}{3}$ & $\frac{1}{6}$& $\frac{1}{6}$ & $\frac{1}{3}$\\
       \midrule
          \begin{minipage}[b]{0.3\columnwidth}
        \centering
        \raisebox{-.5\height}{\includegraphics{figure/E/point2.png}}
    \end{minipage}&
       $\frac{1}{2}$ & $\frac{1}{2}$ &0 & $\frac{1}{6}$\\
       &1&0&0&$\frac{2}{3}$&$\frac{1}{3}$\\
       \bottomrule
    \end{tabular}
    \end{table}
    \begin{table}[H]
        \caption{\textbf{无网格法优化的数值积分方案——四次梯度}}\label{Cfour}
        \centering
        \begin{tabular}{cccccc}
       \toprule
       数值积分点&$\xi$ & $\eta$ & $\gamma$ & $w$ & $w_B$\\
       \midrule
    \begin{minipage}[b]{0.3\columnwidth}
        \centering
        \raisebox{-.5\height}{\includegraphics{figure/E/point.png}}
    \end{minipage}&
    $\eta_a$&$\frac{1-\xi_a}{2}$&$\frac{1-\xi_a}{2}$&$w_a$\\
    &$\eta_b$&$\eta_b$&$\frac{1-\xi_b}{2}$&$w_b$\\
    $\xi_a=0.10810301816870,w_a=0.223381589678011$\\
    $\xi_b=0.816847572980459,w_b=0.109951743655322$\\
    \midrule
    \begin{minipage}[b]{0.3\columnwidth}
        \centering
        \raisebox{-.5\height}{\includegraphics{figure/E/point2.png}}
    \end{minipage}&
    $\frac{1}{3}$&$\frac{1}{3}$&$\frac{1}{3}$&$\frac{9}{20}$\\
    &$1$&$0$&$0$&$-\frac{1}{30}$&$\frac{1}{20}$\\
    &$\frac{1}{2}$&$\frac{1}{2}$&$0$&$\frac{4}{135}$&$\frac{16}{46}$\\
    &$\frac{7+\sqrt{21}}{14}$&$\frac{7+\sqrt{21}}{14}$&$0$&$\frac{49}{540}$&$\frac{49}{180}$\\
    \bottomrule
    \end{tabular}
    \end{table}

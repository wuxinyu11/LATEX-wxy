\chapter{结论与展望}
\section{结论}
本文研究以Hellinger-Reissner变分原理为基础,依托于Hellinger-Reissner原理内嵌本质边界条件的特点,完善了缺乏完备理论基础的再生光滑梯度数值分析方法,为具有变分一致性的无网格分析方法提供了新思路,提出了一种具有高效、鲁棒的具有变分一致性的本质边界施加方案的新方法,能够有效提高传统无网格法的计算精度和效率。
具体结论如下:
\par
(1)以Hellinger-Reissner变分原理为基础,提出了一种满足积分约束条件的变分一致高效本质边界条件施加方法。该方法采用混合离散近似Hellinger-Reissner变分原理弱形式中位移和应力,
其中位移采用传统无网格形函数进行离散,而应力则在每个背景积分域上近似为对应阶次的多项式,满足局部的变分一致性。通过Hellinger-Reissner变分原理完善变分一致型伽辽金无网格数值积分方法的基础理论框架。
\par
(2)在Hellinger-Reissner变分原理的框架下,该方法的离散平衡方程中具有与传统Nitsche法相类似的表达式,可视为与再生光滑梯度积分法相配套的新型Nitsche法。
与传统Nitsche法相比,所提方法的修正变分项采用传统无网格形函数和再生光滑进行混合离散,在保证了变分一致性的同时避免了复杂耗时的形函数导数计算,明显提高了计算效率;
而对于Nitsche法中的稳定项则直接源于Hellinger-Reissner变分原理弱形式中,无需额外增加稳定项,更重要的是稳定项中不包含任何人工参数,有效消除了Nitsche法中的人工参数依赖性。
在该论文中通过对弹性力学问题典型算例和薄板问题典型算例系统的验证了所提方法基于Hellinger-Reissner变分原理施加本质边界条件方法的变分一致性、计算精度和计算效率。
\section{展望}
本文仅是对弹性力学问题和薄板问题的典型算例进行验证所提方法的变分一致性、计算精度和计算效率,在本论文的典型算例中均是属于静力学问题,于是后续的研究工作有以下两点考虑:\par
(1)研究更为复杂的壳体问题基于Hellinger-Reissner变分原理的变分一致型伽辽金无网格法的变分一致性、计算精度和计算效率。本论文着重探讨基于Kirchhoff薄板假设理论进行推导,所以后续可以研究无法忽略剪切变形的壳体问题。\par
(2)研究动力学问题的更多算例基于Hellinger-Reissner变分原理的挠度、散射特性。本论文着重探讨的算例验证均是属于静力学问题,所以后续研究可以探讨关于动力学的算例问题进一步验证所提方法的计算效果。

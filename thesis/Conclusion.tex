% ltex: language=zh-CN
\chapter{结论与展望}
\section{结论}
本文研究以赫林格-赖斯纳变分原理为基础,提出了一种新型的变分一致型伽辽金无网格法。该方法可视为再生光滑梯度无网格数值积分方案与赫林格-赖斯纳原理施加本质边界条件方案相结合,不仅为变分一致型伽辽金无网格法提供了完备的变分理论基础,且建立了一种基于赫林格-赖斯纳的新型本质边界条件施加方案。该方法可以满足全域变分一致性,保证伽辽金法求解精度。且边界条件施加过程自动存在与赫林格-赖斯纳伽辽金弱形式中,施加过程无需人工经验参数即可保证刚度矩阵的正定性要求,整体计算稳定、高效。具体结论如下:

首先,本文针对弹性力学问题建立赫林格-赖斯纳变分一致无网格法。弹性力学问题赫林格-赖斯纳变分原理的变量为应力分量和位移,借助再生光滑梯度积分无网格法的思想,本文所提方法采用混合离散近似相应的应力和位移。其中,位移采用传统的无网格形函数进行离散,而应力则在每个背景积分域内假设为相应阶次的多项式。通过对赫林格-赖斯纳伽辽金弱形式进行整理得到的离散控制方程可视为再生光滑梯度数值积分方案结合新型赫林格-赖斯纳本质边界条件施加方案。再生光滑梯度内嵌积分约束条件,可满足积分域内局部变分一致性条件。而本质边界条件施加过程具有与传统Nitsche法相类似的格式,同样具有一致项和稳定项。所提方法的一致项表达式与Nitsche法的一致项相同,唯一区别是将传统计算复杂的形函数梯度替换为光滑梯度,提升本质边界条件施加过程的计算效率。相较于Nitsche法引入罚函数法作为稳定项,所提方法的稳定项则自然存在于赫林格-赖斯纳伽辽金弱形式中,稳定项中无需人工经验参数,消除人工参数的依赖性。本文采用了一些列弹性力学的经典算例验证本文所提方法的计算精度、效率和误差收敛性,并选用具有变分一致性的再生光滑梯度积分方案和不具有变分一致性的高斯积分方案结合罚函数法、拉格朗日乘子法、Nitsche法作为对比项。结果表明满足全域变分一致性的再生光滑梯度积分方案结合Nitsche法和本文所提方法可以保证计算精度和理论误差收敛率。在效率方面,由于所提方法施加本质边界条件的过程均采用高效的光滑梯度进行计算,所以其计算效率要由于Nitsche法。且所提方法无需额外经验参数,使用的便利性也由于Nitsche法。

其次,本文还将赫林格-赖斯纳变分一致型伽辽金无网格法推广至薄板问题。薄板问题中的赫林格-赖斯纳变分原理的变量为弯矩和挠度,同样地将挠度采用无网格形函数进行离散,弯矩则采用比无网格基函数低两阶的多项式进行离散。整体离散控制方程同样与再生光滑梯度积分方案结合Nitsche法类似,所提方法的优势在薄板问题中更加显著。其中,所提方法一致项中同样采用二阶光滑梯度及其导数代替Nitsche法中传统无网格形函数的二阶及以上导数。值得注意的是Nitsche法的一致项需要计算无网格形函数的三阶导数,而形函数的三阶导数在原本的弱形式当中并不需要。而所提方法采用的二阶光滑梯度具有显示表达式,可直接计算其导数并在一致项中使用,计算效率远由于传统无网格形函数的三阶导数。在稳定项中,Nitsche法仍需要带有人工经验参数的罚函数法稳定计算结果,此时人工经验参数与无网格节点间距相关,不同的无网格离散模型需要采用不同的人工经验参数,不便于使用。同样采用传统薄板典型算例对所提方法进行验证,结果表明所提方法在不需要人工经验参数的前提下,即可得到与Nitsche法采用最优人工经验参数相当的计算结果。在计算效率方面,二阶光滑梯度的直接导数计算效率高于传统形函数的三阶导数,所提方法在计算施加本质边界所需要的形函数耗时远低于Nitsche法。

最后,本文采用了实际工程算例——薄板形抗震阻尼器数值仿真分析,对赫林格-赖斯纳变分一致型伽辽金无网格法进行验证。在工程算例中采用了3种不同形状的抗震阻尼器进行分析对比,相应的材料均假设为弹线性模型。结果表明,本文所提方法对复杂几何形状的问题进行分析时,展现了更加稳定的应力分析云图,比传统罚函数法、Nitsche法更高的鲁棒性和稳定性。所提方法能成为实际工程中实体和薄板模型的一种稳定、可靠和高效的数值仿真工具。

\section{展望}
基于赫林格-赖斯纳原理的变分一致型伽辽金无网格法可以满足全域的变分一致性计算稳定、效率高,且计算过程需要额外人工经验参数使用便利。虽然本文已采用实际工程算例对本方法进行验证,但离真正的应用还差距。后续针对本文所提方法的研究工作可以针对以下几点进行开展:

(1) 发展针对薄壳问题的赫林格-赖斯纳原理变分一致型伽辽金无网格法。与薄板结构类似,薄壳结构同样遵循Kirchhoff假设理论,属于高阶问题。所提方法仅需配合相应的赫林格-赖斯纳原理即可推广至薄壳问题。

(2) 研究动力学问题的赫林格-赖斯纳原理变分一致性型伽辽金无网格法。本论文探究的问题皆为静力问题,本方法可与不同的时域积分方法相结合研究其在动力问题中的有效性,并系统讨论其计算精度、效率和稳定性。

(3) 本文且考虑线弹性、小变形的材料模型,后续研究可将所提赫林格-赖斯纳原理变分一致性伽辽金无网格法推广至材料非线性和大变形情况,以将该方法应用到结构损伤破坏分析当中。

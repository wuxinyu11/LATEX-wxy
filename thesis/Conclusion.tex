\chapter{结论与展望}
\section{结论}
本文研究以Hellinger-Reissner变分原理为基础,依托于Hellinger-Reissner变分原理内嵌本质边界条件的特点,建立具有变分一致性且不依赖于人工经验参数的本质边界条件施加方法,具体结论如下:
\par
(1)本文以Hellinger-Reissner变分原理为基础,提出了一种满足积分约束条件的变分一致高效本质边界条件施加方法。在弹性力学问题中,该方法采用混合离散近似Hellinger-Reissner变分原理弱形式中位移和应力, 其中位移采用传统无网格形函数进行离散,而应力则在每个背景积分单元上近似为对应阶次的多项式。
在Hellinger-Reissner变分原理的框架下, 该方法的离散平衡方程具有与传统Nitsche法相类似的格式,可视为与再生光滑梯度积分法相配套的新型Nitsche法。与传统Nitsche法相比,所提方法的修正变分项采用无网格形函数和再生光滑梯度进行混合离散,在保证了变分一致性的同时避免了复杂耗时的形函数导数计算,明显提高了计算效率;
而对应于Nitsche法中的稳定项则直接源于Hellinger-Reissner变分原理弱形式,无需额外增加稳定项,更重要的是稳定项中不包含任何人工参数,有效消除了Nitsche法中的人工参数依赖性问题。通过典型算例系统地验证了所提基于Hellinger-Reissner变分原理施加本质边界条件方法的变分一致性、计算精度和计算效率。\par
(2)在薄板问题中发展了一种基于Hellinger-Reissner变分原理满足积分约束条件的高效本质边界条件施加方法。该方法基于Hellinger-Reissner变分原理,其中挠度采用传统再生核无网格近似,弯矩则通过在背景积分域中以相应阶次的局部多项式近似离散的混合离散方式,
在Hellinger-Reissner变分原理的框架下, 该方法的离散平衡方程具有与传统Nitsche法相类似的格式,可视为与再生光滑梯度积分法相配套的新型Nitsche法。与传统Nitsche法相比,所提方法的修正变分项采用无网格形函数和再生光滑梯度进行混合离散,在保证了变分一致性的同时避免了复杂耗时的形函数导数计算,明显提高了计算效率;
而对应于Nitsche法中的稳定项则直接源于Hellinger-Reissner变分原理弱形式,无需额外增加稳定项,更重要的是稳定项中不包含任何人工参数,有效消除了Nitsche法中的人工参数依赖性问题。通过典型算例系统地验证了所提基于Hellinger-Reissner变分原理施加本质边界条件方法的变分一致性、计算精度和计算效率。
\section{展望}

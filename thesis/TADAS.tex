\chapter{工程实际问题}
在本章中对三角板减振刚度阻尼器施加受力,
\section{TADAS阻尼器}

在建筑和工程结构中,振动是一个常见的问题,其可能会导致结构的疲劳破坏等问题,传统方法中通过采用增加结构的刚度或使用液体阻尼器、摩擦阻尼器减小结构的振动响应,然而在传统方法中或多或少的存在有效性不高,经济适用性低等问题。
为了克服传统方法的限制,三角板减振刚度阻尼器被引入(TADAS),TADAS阻尼器是一种基于能量耗散原理的被动控制装置,通过在结构中引入附加的阻尼力来吸收和耗散结构的振动能量,能够有效地减小结构地的振动幅值和振动周期,从而显著改善结构的振动响应,
并且TADAS阻尼器的设计相对简单,通常由一块或多块金属材料制成,安装简易、价格低廉,是一种在结构工程中广泛应用于减震和控制结构的被动控制装置。\par
图(\ref{TADAS1})为一个带有TADAS阻尼器的实验装置\textsuperscript{\cite{mohammadi2017,kim2016}},为常在道路、住房和城市中心建造的一层框架大比例模型,
该框架高3米,跨度4米,框架柱采用标准的双IPE180型钢材,梁的工字截面由三块4000*200*12mm的钢板连续焊接而成。支撑体系统一采用双100*100*20mm角度,柱基座使用销连接。
如图(\ref{TADAS2})所示,TADAS阻尼器中的三角形板的上端设为简支固定,下端施加$P=100000$的力。三角形钢板的材料系数为杨氏模量$E=2\times 10^{11}$、泊松比$\nu=0.3$。
\begin{figure}[H]
    \centering
    \includegraphics[scale=0.4]{figure/TADAS/1.png}
    \caption{实验装置示意图}\label{TADAS1}
\end{figure}
\begin{figure}[H]
    \centering
    \begin{subcaptiongroup}
            \includegraphics[width=0.69\textwidth]{figure/TADAS/2.png}
            \phantomcaption\label{TADAS2}
            \includegraphics[width=0.29\textwidth]{figure/TADAS/3.png}
            \phantomcaption\label{TADAS3}
            \end{subcaptiongroup}
        \caption{TADAS阻尼器示意图:\subref{TADAS2} 钢板焊接TADAS装置详图;\subref{TADAS3} 三角形钢板横截面图}
    \label{TADAS2}
\end{figure}
如图()所示,三角形板通过节点进行离散。图()为TADAS阻尼器三角形板的弯矩应力云图,从图中可以看出



\begin{abstract}
无网格法是结构数值仿真领域中一类新型分析方法,该方法依据离散节点信息直接在全域上建立高阶光滑的形函数,适用于结构大变形问题、极端破坏问题和薄板高阶问题。
在各类无网格法中,变分一致型伽辽金无网格法计算稳定高效,可保证伽辽金法理论误差收敛性。该方法通常以假定应变为基础构造积分域上的光滑梯度以构造变分一致无网格法数值积分方案,满足积分域内局部变分一致性,解决无网格法数值积分不稳定问题,保证伽辽金无网格法的计算精度。同时,光滑梯度替代了复杂耗时的传统形函数导数,采用传统低阶高斯积分即可适用于伽辽金法求解过程,提高了伽辽金无网格法的计算效率。
然而,目前一致型伽辽金无网格法缺乏统一的变分理论基础,以进行该类方法的理论误差估计。并且一致型无网格数值积分方案需要配合一致型本质边界条件施加方法,以满足全域的变分一致性。
最常使用的传统 Nitsche 法本质边界条件施加方案需要引入罚函数法作为稳定项,通过设定人工经验参数以保证计算结果的稳定性。此外,高阶问题 Nitsche 法中一致项需要计算无网格形函数高阶梯度以满足变分一致性,但无网格形函数高阶导数形式复杂、计算效率低。

本文针对变分一致型伽辽金无网格法的变分基础和本质边界条件施加方案存在的问题,提出了一种基于弹性力学问题和薄板问题赫林格-赖斯纳变分原理的新型变分一致型伽辽金无网格法。
赫林格-赖斯纳变分原理以应力为变量的余能泛函为基础,余能泛函中包含本质边界条件,无需额外施加本质边界条件。而外力边界条件则通过拉格朗日乘子法进行施加,其中拉格朗日乘子为结构的位移。
位移和应力采用混合离散进行近似,其中位移采用传统无网格形函数进行离散,应力则参考假定应变理论在背景积分域内假设为相应阶次的多项式。
此时,相对应的离散控制方程具有与传统伽辽金无网格法结合 Nitsche 法施加本质边界条件相类似的格式。其中,伽辽金弱形式中的应力由光滑梯度与位移节点系数组成,光滑梯度满足背景积分域内的局部变分一致性。
本质边界条件施加方案中也包含一致项和稳定项,一致项表达式与 Nitsche 法中一致项相似,唯一区别是 Nitsche 法中无网格形函数的高阶梯度采用光滑梯度及其导数替换。特别是薄板问题中,传统 Nitsche 法需要额外计算无网格形函数的三阶导数,而所提方法则替换成二阶光滑梯度的导数,提高了施加本质边界条件过程中的计算效率。
稳定项则自然存在于赫林格-赖斯纳伽辽金弱形式中,无需额外施加稳定项即可保证数值结果的稳定性。且稳定项不包含人工经验参数,消除人工参数的依赖性,更加便于使用。
最后,本文采用了弹性力学问题和薄板问题的系列经典算例验证了所提赫林格-赖斯纳变分一致型无网格法的计算精度、效率和稳定性。并将该方法运用到薄板型抗震阻尼器数值仿真分析中,验证所提方法的有效性。
\end{abstract}
\keywords{无网格法;赫林格-赖斯纳变分原理;本质边界条件;再生光滑梯度;变分一致性}
\begin{abstractEn}
Meshfree methods are a class of emerging numerical formulations for structure analysis. 
These methods discretize the structure directly into nodes and construct high order smoothed shape functions across the entire domain. This makes them well-suited for problems involving large deformation, extreme failures and thin plate.
Among various meshfree methods, the variationally consistent Galerkin method demonstrates superior performance in terms of stability and efficiency. It can guarantee the optimal error convergence rate inherent to the Galerkin method. 
This method typically employs the assumed strain method to construct smoothed gradients within each integration cell. This scheme enforces local variational consistency, overcomes the instability issue associated with traditional integration, and ensures the accuracy of the Galerkin formulation. 
Additionally, by replacing the non-polynomial derivatives of shape functions with polynomial smoothed gradients, the variationally consistent meshfree method allows for the efficient application of the traditional Gauss integration rule within the Galerkin weak form, further enhancing computational efficiency.
However, current variationally consistent meshfree formulations lack a unified variational foundation for theoretical error estimates. 
Furthermore, consistent meshfree numerical schemes require cooperation with consistent essential boundary condition enforcement to satisfy global variational consistency. The commonly used Nitsche's method for enforcing essential boundary conditions relies on the penalty method with a large artificial parameter to achieve stability, which introduces additional complexities. Additionally, the consistent term in Nitsche's method for high-order problems requires calculating the high-order gradient of the meshfree shape functions, which is both complex and computationally expensive.

This thesis proposed a novel variational consistent meshfree method based on the Hellinger-Reissner variational principle for elasticity and thin plate problems. 
This method addressed the limitations of current approaches by establishing a robust variational foundation for the consistent Galerkin meshfree formulation. 
The Hellinger-Reissner principle utilizes a complementary energy functional with stress as the variable. This functional inherently incorporates essential boundary conditions, eliminating the need for additional enforcement. 
External force boundary conditions are imposed using the Lagrange multiplier method, where the Lagrange multiplier is represented by the displacement.
A mixed formulation was adopted for approximating displacement and stress. The displacement is discretized using traditional meshfree shape functions. Similar to the assumed strain theory, the stress is assumed to be a polynomial of corresponding order within the background integration cells. The resulting discrete governing equation shares similarities with the traditional Galerkin meshfree method employing Nitsche's method for essential boundary conditions. In the proposed method's Galerkin weak form, stresses were evaluated using smoothed gradients and displacement nodal coefficients. The smoothed gradients guarantee local variational consistency within the background integration cells.
The essential boundary condition enforcement scheme also included consistent and stabilization terms. The consistent term was similar to that in Nitsche's method, with the key difference being the replacement of high-order derivatives from traditional meshfree shape functions with smoothed gradients and their direct derivatives. Notably, for thin plate problems, the proposed method replaces the calculation of third-order derivatives of meshfree shape functions (required by Nitsche's method) with derivatives of the second-order smoothed gradients, significantly improving the computational efficiency of essential boundary condition enforcement.
The stabilization term naturally emerges within the Hellinger-Reissner Galerkin weak form, eliminating the need for additional stabilization terms to ensure numerical stability. Furthermore, the absence of artificial parameters in the stabilization term simplifies usage and eliminates parameter dependence.
Finally, a series of classic numerical examples for elasticity and thin plate problems are employed to verify the proposed Hellinger-Reissner variationally consistent meshfree method's computational accuracy, efficiency, and stability. Additionally, the method is applied to the numerical simulation analysis of thin plate seismic dampers, demonstrating its effectiveness.
\end{abstractEn}
\keywordsEn{Meshfree method;Hellinger-Reissner variational principle;Essential boundary condition;Reproducing kernel gradient smoothing;Variational consistency}

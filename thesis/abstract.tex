\begin{abstract}
无网格法是结构数值仿真领域中一类新型分析方法,该方法依据离散节点信息直接在全域上建立高阶光滑的形函数,适用于结构大变形问题、极端破坏问题和薄板高阶问题。
在各类无网格法中,变分一致型伽辽金无网格法计算稳定高效,可保证伽辽金法理论误差收敛性。该方法通常以假定应变为基础构造积分域上的光滑梯度以构造变分一致无网格法数值积分方案,满足积分域内局部变分一致性,解决无网格法数值积分不稳定问题,保证伽辽金无网格法的计算精度。同时,光滑梯度替代了复杂耗时的传统形函数导数,采用传统低阶高斯积分即可适用于伽辽金法求解过程,提高了伽辽金无网格法的计算效率。
然而,目前一致型伽辽金无网格法缺乏统一的变分理论基础,以进行该类方法的理论误差估计。并且一致型无网格数值积分方案需要配合一致型本质边界条件施加方法,以满足全域的变分一致性。
最常使用的传统 Nitsche 法本质边界条件施加方案需要引入罚函数法作为稳定项,通过设定人工经验参数以保证计算结果的稳定性。此外,高阶问题 Nitsche 法中一致项需要计算无网格形函数高阶梯度以满足变分一致性,但无网格形函数高阶导数形式复杂、计算效率低。

本文针对变分一致型伽辽金无网格法的变分基础和本质边界条件施加方案存在的问题,提出了一种基于弹性力学问题和薄板问题赫林格-赖斯纳变分原理的新型变分一致型伽辽金无网格法。
赫林格-赖斯纳变分原理以应力为变量的余能泛函为基础,余能泛函中包含本质边界条件,无需额外施加本质边界条件。而外力边界条件则通过拉格朗日乘子法进行施加,其中拉格朗日乘子为结构的位移。
位移和应力采用混合离散进行近似,其中位移采用传统无网格形函数进行离散,应力则参考假定应变理论在背景积分域内假设为相应阶次的多项式。
此时,相对应的离散控制方程具有与传统伽辽金无网格法结合 Nitsche 法施加本质边界条件相类似的格式。其中,伽辽金弱形式中的应力由光滑梯度与位移节点系数组成,光滑梯度满足背景积分域内的局部变分一致性。
本质边界条件施加方案中也包含一致项和稳定项,一致项表达式与 Nitsche 法中一致项相似,唯一区别是 Nitsche 法中无网格形函数的高阶梯度采用光滑梯度及其导数替换。特别是薄板问题中,传统 Nitsche 法需要额外计算无网格形函数的三阶导数,而所提方法则替换成二阶光滑梯度的导数,提高了施加本质边界条件过程中的计算效率。
稳定项则自然存在于赫林格-赖斯纳伽辽金弱形式中,无需额外施加稳定项即可保证数值结果的稳定性。且稳定项不包含人工经验参数,消除人工参数的依赖性,更加便于使用。
最后,本文采用了弹性力学问题和薄板问题的系列经典算例验证了所提赫林格-赖斯纳变分一致型无网格法的计算精度、效率和稳定性。并将该方法运用到薄板型抗震阻尼器数值仿真分析中,验证所提方法的有效性。
\end{abstract}
\keywords{无网格法;Hellinger-Reissner变分原理;本质边界条件;再生光滑梯度;变分一致性}
\begin{abstractEn}
Meshfree methods are a class of emerging numerical formulations for structure analysis. 
These methods discretize the structure directly into nodes and construct high order smoothed shape functions, and thus, they are suitable for large deformation problem, extreme failure problem, high order thin plate problem, etc.
Among various meshfree methods, the variationally consistent Galerkin meshfree method shows high performance in stability and efficiency, and it can guarantee the optimized error convergence rate of Galerkin method.
This method typically constructs smoothed gradients in each integration cells using assumed strain method. This scheme can satisfies the local variational consistency, addresses the integration issue of instability and ensures the accuracy of Galerkin formulation.
Meantime, the traditional efficient Gauss integration rule can be used in Galerkin weak form due to the non-polynomial derivatives of shape functions were replaced using polynomial smoothed gradients, which can further improve the efficiency.
However, current variationally consistent meshfree formulations lack a unified variational foundation for theoretical error estimates.
Moreover, the consistent meshfree numerical scheme also needs to cooperate with a consistent essential boundary condition enforcement to satisfy the global variational consistency.
The traditional Nitsche's method, which is the most commonly used essential boundary condition enforcement, needs to introduce the penalty method with a large enough artificial parameter is employed as a stabilization term for ensuring the stability of results.
In addition, the consistent term of Nitsche's method in high order problem needs to compute the high order gradient of the meshfree shape functions, but the high order derivatives of meshfree shape functions are complex and also inefficient.

This thesis proposed a novel variational consistent Galerkin meshfree method based on the Hellinger-Reissner variational principle for elasticity and thin plate problems. This method established the variational foundation for consistent Galerkin meshfree formulation and can address the issues in current essential boundary condition enforcement.
Firstly, the Hellinger-Reissner variational principle is based on the complementary energy functional with stress as the variables, while the complementary energy functional includes the essential boundary conditions and the extra imposition of essential boundary can be eliminated.
The external force boundary  condition is imposed by the Lagrangian multiplier method, where the Lagrangian multiplier is chosen as the displacement.
A mixed formulation is used herein for approximation of displacement and stress, where the displacement is discretized by the traditional meshfree shape functions, and, similar with assumed strain theory, the stress is assumed to be a polynomial with corresponding order in the background integration cells.
The corresponding discrete governing equation has a form similar to that of the traditional Galerkin meshfree method with Nitsche's method to impose essential boundary conditions.
In the Galerkin weak form of proposed method, the stresses were evaluated using smoothed gradient and the displacement nodal coefficients, the smoothed gradient satisfied the local variational consistency within the background integration cell.
The essential boundary condition enforcement also includes consistent and stabilization terms. The expression of the consistent term is similar to that in the Nitsche's method, the only difference is that the high order derivatives of traditional meshfree shape function in the Nitsche's method is replaced by the smoothed gradients and their direct derivatives.
Particularly for thin plate problems, the traditional Nitsche's method requires additional calculation of the third order derivatives of the meshfree shape functions, while the proposed method replaces it with the derivative
\end{abstractEn}
\keywordsEn{Meshfree method;Hellinger-Reissner variational principle;essential boundary condition;regenerative smooth gradient;variational consistency}

\chapter{绪论}
\section{引言}
由于实际工程中的结构通常具有复杂的几何形状、材料是非线性以及多样化的荷载情况,从而使得经典的解析方法难以直接应用于工程实际中进行计算和分析,需要借助数值计算分析方法进行求解。
有限元法\cite{hughes2000,2014Computer,stein2018,2013Nonlinear}是目前最常采用的数值分析方法,是通过将复杂的结构问题离散化为许多小的有限元单元,并在每个单元上建立近似的数学模型,进而将复杂的问题转化为求解一系列简化的局部问题。
这种离散化的方法使得有限元法能够有效地处理各种复杂的几何结构模型,有限元法还可以通过划分网格数量,调整网格密度灵活地处理各种几何形状,进一步提高计算精度。
虽然有限元法是一种广泛应用于解决复杂工程问题的数值方法,但有限元的计算结果在很大程度上依赖于网格的划分。对于复杂的薄板几何形状,需要细致划分网格才能获得准确的解,网格密度增加会引起计算增加,降低计算效率。
值得注意的是,基于单元插值的有限元法通常只有$C^0$联系,难以构造整体协调的$C^1$单元,无法直接求解要求$C^1$是连续的薄板等高阶问题。

无网格法\cite{chen2017,belytschkoMeshlessMethodsOverview1996b}是一类能够有效解决高阶薄板问题的数值分析方法,
该方法是根据离散节点位置信息直接建立形函数的方法,其形函数具有高阶光滑的特点,不依赖于网格单元信息构造形函数,有效的避免了网格划分,网格畸变等问题,使得无网格法在处理薄板问题\cite{邓立克2019薄板分析的线性基梯度光滑伽辽金无网格法}中的几何变化具有更好的适应性,稳定性和可靠性。无网格法可以直接考虑薄板的厚度变化,能够准确地捕捉薄板的弯曲和扭转状态。
根据无网格形函数构造不依赖于网格的特点,该方法也适用于其它复杂的几何结构模型的离散,
如大变形分析\cite{陈嵩涛2020几何非线性分析的高效高阶无网格法}、裂纹扩展模拟\cite{GaoXin2018},有效的提高因为网格划分因素所引起的计算精度下降的问题。

\section{伽辽金无网格法研究历史及现状}
上个世纪七十年代,由Lucy\cite{1977A}、Gingold和Monaghan\cite{gingold1977}提出的光滑水动力学法(SPH,Smoothed Particle Hydrodynamics)开启了众多学者对无网格法的关注,该方法是采用核函数的近似方法通过对求解域进行离散化,并使用强形式进行数值求解的近似方法,该方法属于配点型无网格法。
该方法具有构造简单,无需数值积分等优势被广泛应用于进行数值计算分析,但由于配点型无网格法是通过核近似的离散方式一般不满足一致性条件,使得该方法无法准确保证数值计算的稳定性及计算精度\cite{auricchio2010a,wang2018a,wang2020,gomez2016}。

1992年Nayroles等人\cite{nayroles1992}使用移动最小二乘近似形函数并通过伽辽金弱形式提出了散射元法。
1994年Belytschko等人\cite{belytschko1994}指出了散射近似(Scattered Data Approximation)实质上等同于移动最小二乘近似(Moving Least Squares, MLS)。通过对移动最小二乘近似函数进行精确求导,引入背景网格和高阶高斯积分方法改善数值积分的精度,通过引入拉格朗日乘子,可以在伽辽金无单元法中有效的施加边界条件,从而提高计算精度和稳定性。在这些改进的基础上,Belytschko等人将方法命名为伽辽金无单元法(Element Free Galerkin Method, EFG)。
值得注意的是,在采用伽辽金法进行求解的时候,由于无网格形函数一般不是多项式,计算时需要采用建立高阶高斯积分法或其他数值积分方法\cite{de2001,carpinteri2002,WangBingBing2019},从而会导致计算效率降低\cite{melenk1996,WuJunChao2016}。为了提高伽辽金无网格法的计算效率,众多学者进行了相关研究工作致力于解决伽辽金无网格法在计算过程中的效率问题\cite{wang2016,wang2019,chen2001,chen2013,duan2012,duan2014}。

1995年Liu等人\cite{liu1995}在光滑水动力学(Smoothed Particle Hydrodynamics, SPH)核近似方法的基础上进行了改进。他们引入了核近似的多项式再生条件和校正函数,提出了再生核近似(Reproducing Kernel Approximation, RK)和再生核质点法(Reproducing Kernel Particle Method, RKPM)。
1997年Babuška和Melenk\cite{melenk1996,babuska1997}将系数矩阵表示为单位矩阵加上一个低秩矩阵的形式,使得原始线性方程组表示为多个规模较小的子问题的组合,每个子问题可以被独立的求解,进而求解出整个方程组的解,该方法称为单位分解法(PUM, Partition of unity method)也称为广义有限元法\cite{strouboulis2001}。
1998年Sukumar等人\cite{sukumar1998}提出了自然单元法,该方法通过将计算域划分为多个自然单元来离散化问题,通过使用自然领域形函数来近似解,并通过插值技术在整个计算域中进行扩展,该方法的形函数只具有$C^0$连续性,难以求解高阶薄板问题。
为了更好利用无网格法求解薄板问题,
2002年Long S等人\cite{long2002}利用局部加权函数构建近似函数,并在控制方程中使用高斯积分法进行数值积分提出伽辽金局部无网格法(Meshless Local Petrov–Galerkin)。
2006年Liu等人\cite{liu2006}利用径向基函数构建近似函数,并使用Hermite插值获得薄板的位移和旋转的连续性提出了Hermite径向点插值(Hermite Radial Point Interpolation)。
2011年Mill´an D等人\cite{millan2011}利用最大熵原理构建一个最大熵函数,通过最大化信息熵确定最优的近似解提出了最大熵无网格法(Maximum-entropy Meshfree  Method )。
2012年Oh H等人\cite{oh2012}通过将薄板离散为一组粒子,并利用粒子间的相互作用进行计算提出了单元划分方法。
随后2015年程玉民等人\cite{chen2015}提出了复变量(complex variable RKPM )方法,该方法基于粒子离散化,通过复变再生核函数近似求解数值分析。
2016年Thai等人\cite{thai2016}利用移动Kriging插值构建近似函数,使用细化板理论进行描述薄板的受力状态提出了移动Kriging无网格法( Moving Kriging Meshfree Method),进而直接计算出薄板的位移和应力场。
王东东,王莉华等人\cite{wang2020,wang2021}提出了一种无网格配点法,通过利用梯度再生核函数进行构建近似函数,并通过引入梯度项提高数值计算的稳定性和精度,可以准确计算弹性梁和薄板的位移和应力分析,并进行相应的静态和动态分析。
为了更好的利用无网格法解决不同类型的问题,现如今已经提出了各式各样的无网格方法\cite{ChengYuMin2005,TanXianYun2011,LianYanPing2013,ZhangXiong2017,GaoXiaoWei2019},关于更多目前无网格法的研究进展见文献\cite{nguyen2008,liu2009,张雄2009无网格法的理论及应用,wang2014,yreux2017,koester2019,rohit2020,王莉华2021配点型无网格法理论和研究进展,LiuYuXiang2021,朱志辉2021基于,sriram2021,ChenJian2022,LiYuDong2022}

本论文讨论的是具有变分一致性的伽辽金无网格法\cite{babuska2008,wu2021}。
无网格形函数及其梯度通常为有理式,并且形函数的影响域具有高度重叠的特点导致无网格形函数在背景积分单元上为分段的有理式。
在伽辽金法的求解过程中,传统基于多项式完备性建立的高斯积分法无法进行准确数值积分过程,导数伽辽金无网格法无法准确求解与其基函数阶次相同的解析解,即不满足变分一致性\cite{1999Numerical}。
为了解决该问题,建立具有变分一致性的伽辽金无网格法成为无网格研究领域的热门问题。目前具有变分一致性的伽辽金无网格法主要是通过假定应变理论构造匹配的光滑梯度,光滑梯度为低阶多项式,采用低阶高斯积分法既能准确进行数值积分,保证计算误差的收敛性。
同时,光滑梯度的构造过程仅需要计算传统无网格形函数,避免复杂耗时的形函数梯度计算,提高传统伽辽金无网格法的计算效率。
为了满足积分约束条件,Chen等人从伽辽金法的线性准确性条件出发,提出了线性积分约束条件\cite{chen2001}。
段庆林等人\cite{陈嵩涛2020几何非线性分析的高效高阶无网格法,duan2012}将线性积分约束条件推广至高阶情况,并通过高阶积分约束条件计算高斯点处光滑梯度值,使得传统高斯积分法满足变分一致性,该方法称为一致性无单元伽辽金法。
王东东和吴俊超提出了再生光滑梯度理论框架\cite{wang2019},该框架具有与传统无网格形函数相类似的表达式,光滑梯度构造过程中可重新合理优化数值积分点采样点位置和权重,减少无网格形函数的计算量,提高计算效率。
任晓月\cite{wang2023}提出了一致投影积分法,以投影形函数为基础建立替代传统形函数空间,通过将投影形函数代入原始无网格形函数中近似计算伽辽金弱形式,利用具有高阶高斯正交规则的相应阶三角形有限元形函数,进而满足任意阶积分约束条件。

除了以假定应变理论为基础的变分一致型无网格法外,Chen等人\cite{chen1996}提出了修正变分积分法,该方法通过对基函数的变分导数进行近似,并在积分过程中考虑变分一致性,通过选择适当的积分点和积分权重,从而实现任意阶数的数值积分,该方法能够准确的模拟大变形问题、非线性问题和冲击破坏问题,提高伽辽金无网格法中数值积分的准确性和一致性,保证数值计算精度提高计算效率。
但使用伽辽金无网格法求解偏微分方程通常会使用试函数和权函数进行构建离散形式的方程,在伽辽金无网格法中,试函数和权函数通常属于相同的函数空间,从而确保离散形式的刚度矩阵是对称的,而修正变分积分法中修正后的权函数和试函数不属于同一空间,导数刚度矩阵的非对称性,非刚度矩阵会引起数值解的误差,无法保证数值求解的稳定性和精确性。
王东东和吴俊超提出了嵌套子域积分法\cite{wang2016},该方法利用子域划分和梯度平滑技术提高数值积分的准确性,通过将计算域划分为多个子域,并在每个子域内采用光滑应变,利用两层次积分域得到的刚度矩阵进行合理组合,消除二阶误差项进而满足二次变分一致性,但嵌套子域积分法中,确保各层次的嵌套子域完全相似是构造光滑梯度的一个要求,这意味着每个子域在几何形状或大小上都与其他子域完全相同,使得该方法难以推广至三维及高阶情况。

与传统有限元法相比,无网格方法具有高阶连续光滑的特点。然而,这种连续性导致无网格形函数在离散节点上通常不具有插值性,这在求解过程中使得施加本质边界条件变得困难。
为了克服这个问题,许多学者提出了各种具有插值性的无网格近似方法,以便能够直接施加本质边界条件\cite{CaoYang2020,fernandez-mendez2004},如奇异权函数法\cite{kaljevic1997}、插值最小二乘法\cite{liu2019,ChenXinXin2021}、复变量移动最小二乘法\cite{ChengYuMin2005}、广义移动最小二乘法\cite{HuangJuan2007}、变换法\cite{chen2000}等。
然而,这类方法不是建立在变分原理基础上,无法保证节点之间位移边界条件施加精度和无网格法的变分一致性。
对于满足积分约束条件的无网格数值积分方法,如稳定节点积分法\cite{chen2001}、一致性积分法\cite{陈嵩涛2020几何非线性分析的高效高阶无网格法,duan2012}、变分一致积分法\cite{chen2013}、嵌套子域积分法\cite{wang2016}、再生光滑梯度积分法\cite{wang2019}等,在计算过程中采用形函数的光滑梯度替换传统无网格形函数导数,在保证无网格法的计算精度和最优误差收敛率的同时提高了计算效率,但其本质边界条件仍需要具有变分一致性的方法进行施加\cite{WuJunChao2016,hillman2021}。

\section{本文选题背景}
变分一致型伽辽金无网格法可追溯到2004年Chen等人\cite{chen2001}提出的稳定节点积分法,该方法从伽辽金法的线性准确性条件出发,提出了线性积分条件。伽辽金数值积分方法需要从满足积分约束条件,才能求解线性问题,即满足线性的变分一致性。该方法通过在节点周围引入额外的稳定项,改善对梯度突变和奇点的逼近,通过修改形函数和积分权重,构造一种稳定的节点积分方案,节点处的积分权重与节点处的梯度信息相关联,假设光滑梯度为常数,通过满足线性积分约束条件建立光滑梯度实现对数值解的稳定性。尽管稳定节点积分法在提高数值解稳定性和准确性具有一定的优势,但稳定节点积分法需要在节点周围引入额外的稳定项,会降低计算效率,并且额外稳定项的效果很大程度上依赖于参数的选择,选择不当的参数可能会引起数值解的不稳定性。

段庆林等人\cite{陈嵩涛2020几何非线性分析的高效高阶无网格法,duan2012}将线性积分约束条件推广至高阶情况,通过对无网格方法中的形函数进行修正和优化,通过与修正的形函数结合实现对积分的二阶精确性。通过使用高阶RBF函数计算高斯积分点处光滑梯度值,使得传统高斯积分法满足变分一致性,提高几何非线性分析的计算精度和效率。但该一致性积分法在计算过程中选取的高斯点数需要与积分约束条件数保持一致,这导致在处理高阶问题时会降低计算效率。并且一致性积分法无法完全消除数值误差,数值误差在计算高阶导数的过程中会逐渐积累和传播,可能会降低计算精度。

王东东和吴俊超提出了再生光滑梯度理论框架\cite{wang2019},该框架具有与传统无网格形函数相类似的表达式,统一了以假定应变为基础的变分一致型光滑梯度构造方案。在该理论框架下,光滑梯度构造过程可重新合理优化数值积分采样点位置和权重,从而减少无网格形函数的计算量,提高计算效率。
\textbf{目前,再生光滑梯度理论框架为假定应变的光滑梯度提供了一个通用的表达式,并在伽辽金弱形式中直接将光滑梯度替换成传统无网格形函数梯度。但该光滑梯度并不直接等于传统无网格形函数梯度,该过程缺乏完备的变分原理理论基础}。

为了满足全域的变分一致性,满足积分约束条件的无网格数值积分方案需要配合具有变分一致性的本质边界条件施加方案,传统无网格形函数在自身节点处不具有插值性,Belytschko等人\cite{belytschko1994}最早采用拉格朗日乘子法施加本质边界条件,该方法需要引入额外自由度离散拉格朗日乘子,当采用变分一致型无网格数值积分方案时,拉格朗日乘子的自由度需要和光滑梯度构造过程中积分点的位置保持一致,以满足变分一致性。采用过多自由度离散拉格朗日乘子将导致整体刚度矩阵出现奇异,以致于该方法不适合高阶的变分一致型伽辽金无网格法。
罚函数法\cite{zhu1998}施加本质边界条件无需额外增加自由度,数值实现简单。广泛应用于伽辽金无网格法。但该方法的计算误差依赖于人工经验参数,且不具有变分一致性,不能保证计算精度。
Nitshce法\cite{fernandez-mendez2004}是目前变分一致型无网格法主要采用的本质边界条件施加方法,该方法在修正变分原理的基础上引入罚函数法作为保证刚度矩阵的正定性。但是在积分一致的数值积分方案中已经不需要的无网格高阶梯度被重新引入,降低了无网格分析的计算效率。同时,Nitsche法中的稳定性还是需要人工经验参数,过大或过小的人工参数都将导致计算精度的降低。
另一类无网格施加本质边界条件的方法是试图恢复无网格形函数的插值性。Fernández-Méndez与Huerta\cite{fernandez-mendez2004}通过修改无网格形函数构造过程中核函数的权重,使得无网格形函数在边界处具有插值性,但该方法无法满足积分约束条件。Hillman和Lin\cite{hillman2021}在修正变分法中引入具有插值性无网格近似,但该方法改变了解的空间。Chen等人\cite{chen1996}采用转换矩阵,将无网格法中的节点系数重新与物理值建立联系,从而直接施加本质边界条件,并称该方法为变换法。王东东\cite{wang2015}将变换法引入稳定节点积分法中,并对其进行了修正,以保证数值积分的一致性。然而,变换法中转换矩阵需作用于整体刚度矩阵,计算量大,不适用于大规模计算。
\textbf{Nitsche法作为最适合变分一致型伽辽金无网格法的本质边界条件施加方法还存在诸多问题,如人工参数的依赖性,需要计算复杂耗时的形函数高阶梯度等,亟待发展一种全新变分一致型本质边界条件施加方案}。

\section{本文主要内容}
论文研究将发展基于赫林格-赖斯纳原理的变分一致型伽辽金无网格法,并依托于赫林格-赖斯纳原理内嵌本质边界条件的特点,建立具有变分一致性且不依赖于人工经验参数的本质边界条件施加方法,具体内容如下:

(1)\textbf{通过赫林格-赖斯纳变分原理完善变分一致型伽辽金无网格数值积分方法的基础理论框架}。首先,基于赫林格-赖斯纳变分原理推导余能泛函,分别对位移和应力变分可以得到赫林格-赖斯纳原理弱形式。其中位移采用传统无网格形函数进行离散,而应力采用在每个积分域中假设为多项式,以满足局部的变分一致性。最后,通过分片实验验证该方法的变分一致性;

(2)\textbf{以赫林格-赖斯纳原理为基础建立具有变分一致性且不依赖人工参数的本质边界条件施加方法}。首先,HR变分原理弱形式中内嵌本质边界条件,以赫林格-赖斯纳变分原理弱形式为基础系统推导本质边界条件施加过程中的离散控制方程,详细分析该方法的变分一致性。并详细对比所提方法和传统Nitsche法,罚函数法,拉格朗日乘子法之间的差别,并通过传统弹性力学问题和薄板问题验证所提方法的计算精度和效率,最后采用实际工程算例-薄板型抗震阻尼器数值仿真分析,对赫林格-赖斯纳变分一致型伽辽金无网格法进行验证。




\chapter{绪论}
\section{引言}
由于实际工程中的结构通常具有复杂的几何形状、非线性材料行为以及多样化的荷载情况,导致经典的解析方法难以直接应用于计算和分析。
有限元法[]通过将复杂的结构问题离散化为许多小的有限元单元,并在每个单元上建立近似的数学模型,将复杂的问题转化为求解一系列简化的局部问题。这种离散化的方法使得有限元法能够有效地处理复杂的结构造型、各向异性材料特性以及不规则荷载。
通过有限元法,可以灵活地划分网格并调整网格密度,以适应不同区域的复杂性。
有限元法提供了一个强大的工具,可以对结构进行力学分析、优化设计以及可靠性评估。它已广泛应用于建筑、桥梁、飞机、汽车、石油和天然气管道等工程领域[],为工程师和设计师提供了一种可靠、高效的方法来解决实际工程中的复杂问题。
有限元法[]作为一种广泛应用于解决复杂工程问题的数值方法,也仍存在网格划分复杂,网格的划分会极大的影响数值解的准确性和计算效率,在某些情况下,有限元分析对病态的元素会导致数值不稳定的现象,使得解失去可靠性和准确性,不同的网格划分会导致不同的解存在着网格依赖性。\par
基于单元插值的有限元法通常只有$C^0$连续,难以构造整体协调的$C^1$单元,根据Kirchhoff板建设原理[],忽略剪切变形可以得到薄板的四阶控制方程,这一四阶问题要求$C^1$是连续的,这在传统有限元方法是不容易实现的[]。以节点为近似特征的无网格方法[x-xx]为构造任意阶光滑形函数提供了一种非常直接的方法。无网格法[]是一类根据离散节点位置信息直接建立形函数的方法,其形函数具有高阶光滑的特点,适用于薄板等高阶问题。形函数构造过程中不依赖网格信息,适用于复杂区域的离散,如大变形分析[]、裂纹扩展模拟[]能及薄板等高阶问题[],够有效减轻网格畸变所引起的计算精度下降的问题。
\section{伽辽金无网格法研究历史及现状}
1994年Belytschko等人\textsuperscript{\cite{belytschkoElementFreeGalerkin1994}}指出了散射近似(Scattered Data Approximation)实质上等同于移动最小二乘近似(Moving Least Squares, MLS)。通过对移动最小二乘近似函数进行精确求导,引入背景网格和高阶高斯积分方法改善数值积分的精度,通过引入拉格朗日乘子,可以在伽辽金无单元法中有效地施加边界条件,从而提高计算精度和稳定性。
在这些改进的基础上,Belytschko等人将方法命名为伽辽金无单元法(Element Free Galerkin Method, EFG)。1995年Liu等人[]在光滑粒子流体动力学(Smoothed Particle Hydrodynamics, SPH)核近似方法的基础上进行了改进。他们引入了核近似的多项式再生条件和校正函数,提出了再生核近似(Reproducing Kernel Approximation, RK)和再生核质点法(Reproducing Kernel Particle Method, RKPM)。
2002年Long S等人\cite{wuJiYuHeLinGeLaiSiNaBianFenYuanLiDeYiZhiGaoXiaoWuWangGeBenZhiBianJieTiaoJianShiJiaFangFa2022}利用局部加权函数构建近似函数,并在控制方程中使用高斯积分法进行数值积分提出伽辽金局部无网格法(Meshless Local Petrov–Galerkin)。
2006年Liu等人\cite{liuReproducingKernelParticle1995}利用径向基函数构建近似函数,并使用Hermite插值获得薄板的位移和旋转的连续性提出了Hermite径向点插值(Hermite Radial Point Interpolation)。
2011年Mill´an D等人[]利用最大熵原理构建一个最大熵函数,通过最大化信息熵确定最优的近似解提出了最大熵无网格法(Maximum-entropy Meshfree  Method ),通过这种方式从散点数据中生成连续的薄壳几何形状和应力场。
2012年Davis等人[]通过将薄板离散为一组粒子,并利用粒子间的相互作用进行计算提出了单元划分方法。随后2015年Chen等人[]提出了复变量(complex variable RKPM )方法,该方法基于粒子离散化,通过复变再生核函数近似求解数值分析。
2016年Thai等人[]利用移动Kriging插值构建近似函数,使用细化板理论进行描述薄板的受力状态提出了移动Kriging无网格法( Moving Kriging Meshfree Method),进而直接计算出薄板的位移和应力场。
Wang等人[]提出了一种无网格配置方法该方法利用梯度再生核函数进行构建近似函数,并通过引入梯度项提高数值计算的稳定性和精度,可以准确得计算薄弹性梁和薄板的位移和应力分析,并进行相应的静态和动态分析。\par
无网格形函数通常为有理式,并且形函数的影响域具有高度重叠的特点导致无网格形函数在背景积分单元上为分段的有理式。在伽辽金法的求解过程中,传统高斯积分法无法精确数值积分由形函数组成的高度矩阵和力向量,导致伽辽金弱形式不满足积分约束条件,无法保证计算精度和最优误差收敛率。
为了满足积分约束条件,Chen等人从伽辽金法的线性准确性条件出发,提出了线性积分约束条件[]。
段庆林等人[]将线性积分约束条件推广至高阶情况,并通过高阶积分约束条件计算高斯点处光滑梯度值,使得传统高斯积分法满足变分一致性,该方法称为一致性无单元伽辽金法。
随后Chen等人[]提出了修正变分积分法,通过在伽辽金弱形式中采用不同形式权函数和试函数以满足高阶变分一致性,该方法称为变分一致积分法。
王东东和吴俊超[]利用两层次积分域得到的刚度矩阵进行合理组合,消除二阶误差项以满足二次变分一致性,提出了嵌套子域积分法。
王东东和吴俊超提出了再生光滑梯度理论框架[],该框架具有与传统无网格形函数相类似的表达式,光滑梯度构造过程中可重新合理优化数值积分点采样点位置和权重,减少无网格形函数的计算量,提高计算效率。
(Wang等人[]提出了一致投影积分法???)。\par
与传统有限元法相比,无网格方法具有高阶连续光滑的特点。然而,这种连续性导致无网格形函数在离散节点上通常不具有插值性,这在求解过程中使得施加本质边界条件变得困难。
为了克服这个问题,许多学者提出了各种具有插值性的无网格近似方法,以便能够直接施加本质边界条件[],如奇异权函数法[]、插值最小二乘法[]、复变量移动最小二乘法[]、广义移动最小二乘法[]、变换法[]等。
然而,这类方法不是建立在变分原理基础上,无法保证节点之间位移边界条件施加精度和无网格法的变分一致性。在伽辽金无网格法中,拉格朗日乘子法[]是常采用的一种变分一致本质边界条件施加方法,该方法通过在整体刚度矩阵上引入额外的自由度离散拉格朗日乘子施加约束条件满足边界条件。
Liu等人[]根据拉格朗日乘子的物理意义,通过采用位移自由度离散拉格朗日乘子提出修正变分原理法。
为了满足正定性,Zhu等人[]在无网格法引入罚因子施加本质边界中,提出了罚函数法,具有简洁高效的特点,但该方法不满足变分一致性。
Fernández Méndez和Huerta[]提出了Nitsche法施加本质边界条件,通过修正变分项满足变分一致性,引入的罚因子又满足刚度矩阵的正定性,提高求解精度,是目前最常采用的变分一致性本质边界条件施加方法。
\section{本文选题背景}
变分一致型伽辽金无网格法可追溯到2004年Chen等人[]提出的稳定节点积分法,该方法从伽辽金法的线性准确性条件出发,提出了线性积分条件。伽辽金数值积分方法需要从满足积分约束条件,才能秋娥及线性问题,即满足线性的变分一致性。该方法通过在节点周围引入额外的稳定项,改善对梯度突变和奇点的逼近,通过修改形函数和积分权重,构造一种稳定的节点积分方案,节点处的积分权重与节点处的梯度信息相关联,假设光滑梯度为常数,通过满足线性积分约束条件建立光滑梯度实现对数值解的稳定性。尽管稳定节点积分法在提高数值解稳定性和准确性具有一定的优势,但稳定节点积分法需要在节点周围引入额外的稳定项,会降低计算效率,并且额外稳定项的效果很大程度上依赖于参数的选择,选择不当的参数可能会引起数值解的不稳定性。\par
段庆林等人[]将线性积分约束条件推广至高阶情况,通过对无网格方法中的形函数进行修正和优化,通过与修正的形函数结合实现对积分的二阶精确性。通过使用高阶RBF函数计算高斯积分点处光滑梯度值,使得传统高斯积分法满足变分一致性,提高几何非线性分析的计算精度和效率。但该一致性积分法在计算过程中选取的高斯点数需要与积分约束条件数保持一致,这导致在处理高阶问题时会降低计算效率。并且一致性积分法无法完全消除数值误差,数值误差在计算高阶导数的过程中会逐渐积累和传播,可能会降低计算精度。\par
王东东和吴俊超提出了再生光滑梯度理论框架[],该框架具有与传统无网格形函数相类似的表达式,统一了以假定应变为基础的变分一致型光滑梯度构造方案。在该理论框架下,光滑梯度构造过程可重新合理优化数值积分采样点位置和权重,从而减少无网格形函数的计算量,提高计算效率。\par
除了以假定应变理论为基础的变分一致型无网格法外,Chen等人[]提出了修正变分积分法,该方法通过对基函数的变分导数进行近似,并在积分过程中考虑变分一致性,通过选择适当的积分点和积分权重,从而实现任意阶数的数值积分该方法能够准确的模拟大变形问题[]、非线性问题[]和冲击破坏问题[],提高伽辽金无网格法中数值积分的准确性和一致性,保证数值计算精度提高计算效率。但使用伽辽金无网格法求解偏微分方程通常会使用试函数和权函数进行构建离散形式的方程,在伽辽金无网格法中,试函数和权函数通常属于相同的函数空间,从而确保离散形式的刚度矩阵是对称的,而修正变分积分法中修正后的权函数和试函数不属于同一空间,导数刚度矩阵的非对称性,非刚度矩阵会引起数值解的误差,无法保证数值求解的稳定性和精确性。\par
王东东和吴俊超[]提出了嵌套子域积分法,该方法利用子域划分和梯度平滑技术提高数值积分的准确性,通过将计算域划分为多个子域,并在每个子域内采用光滑应变,利用两层次积分域得到的刚度矩阵进行合理组合,消除二阶误差项进而满足二次变分一致性,但嵌套子域积分法中,确保各层次的嵌套子域完全相似是构造光滑梯度的一个要求,这意味着每个子域在几何形状或大小上都与其他子域完全相同,使得该方法难以推广至三维及高阶情况。
目前,再生光滑梯度理论框架为假定应变的光滑梯度提供了一个通用的表达式,并在伽辽金弱形式中直接将光滑梯度替换成传统无网格形函数梯度。但该光滑梯度并不直接等于传统无网格形函数梯度,该过程缺乏完备的变分原理理论基础。\par
为了满足全域的变分一致性,满足积分约束条件的无网格数值积分方案需要配合具有变分一致性的本质边界条件施加方案,传统无网格形函数在自身节点处不具有插值性,Belytschko等人[21]最早采用拉格朗日乘子法施加本质边界条件,该方法需要引入额外自由度离散拉格朗日乘子,当采用变分一致型无网格数值积分方案时,拉格朗日乘子的自由度需要和光滑梯度构造过程中积分点的位置保持一致,以满足变分一致性。采用过多自由度离散拉格朗日乘子将导致整体刚度矩阵出现奇异,以致于该方法不适合高阶的变分一致型伽辽金无网格法。
罚函数法[22]施加本质边界条件无需额外增加自由度,数值实现简单。广泛应用于伽辽金无网格法。但该方法的计算误差依赖于人工经验参数,且不具有变分一致性,不能保证计算精度。
Nitshce法[23]是目前变分一致型无网格法主要采用的本质边界条件施加方法,该方法在修正变分原理的基础上引入罚函数法作为保证刚度矩阵的正定性。但是在积分一致的数值积分方案中已经不需要的无网格高阶梯度被重新引入,降低了无网格分析的计算效率。同时,Nitsche法中的稳定性还是需要人工经验参数,过大或过小的人工参数都将导致计算精度的降低。
另一类无网格施加本质边界条件的方法是试图恢复无网格形函数的插值性。Fernández-Méndez与Huerta[23]通过修改无网格形函数构造过程中核函数的权重,使得无网格形函数在边界处具有插值性,但该方法无法满足积分约束条件。Hillman和Lin[25]在修正变分法中引入具有插值性无网格近似,但该方法改变了解的空间。Chen等人[16]采用转换矩阵,将无网格法中的节点系数重新与物理值建立联系,从而直接施加本质边界条件,并称该方法为变换法。王东东[24]将变换法引入稳定节点积分法中,并对其进行了修正,以保证数值积分的一致性。然而,变换法中转换矩阵需作用于整体刚度矩阵,计算量大,不适用于大规模计算。
Nitsche法作为最适合变分一致型伽辽金无网格法的本质边界条件施加方法还存在诸多问题,如人工参数的依赖性,需要计算复杂耗时的形函数高阶梯度等,亟待发展一种全新变分一致型本质边界条件施加方案。
\section{本文主要内容}
本文研究将发展基于Hellinger-Reissner原理的变分一致型伽辽金无网格法,依托于Hellinger-Reissner原理内嵌本质边界条件的特点,通过对Hellinger-Reissner变分原理弱形式中的变量采用再生核近似与在每个背景积分域中假设为多项式进行混合离散的形式,
建立具有变分一致性且不依赖于人工经验参数的本质边界条件施加方法,提高计算精度,用再生光滑梯度替换传统无网格形函数梯度,减少无网格形函数的计算量,有效提高计算效率。
具体内容如下:\par
(1)基于弹性力学问题,研究对位移和应力采用混合离散的Hellinger-Reissner原理的变分一致型伽辽金无网格法,推导出具有变分一致性的无网格本质边界条件施加方案,通过分片实验和典型弹性力学问题进一步验证该方法的变分一致性,并详细对比所提方法与传统本质边界条件施加方案之间的区别,验证所提方法的计算精度和效率。\par
(2)基于高阶薄板问题,研究对挠度和弯矩采用混合离散的Hellinger-Reissner原理的变分一致型伽辽金无网格法,推导出具有变分一致性的无网格本质边界条件施加方案,通过分片实验和薄板问题进一步验证该方法的变分一致性,并详细对比所提方法与传统本质边界条件施加方案之间的区别,验证所提方法的计算精度和效率。\par

% (3)基于薄板动力分析问题,\par




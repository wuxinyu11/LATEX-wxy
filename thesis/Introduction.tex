\chapter{绪论}
\section{引言}
由于实际工程中的结构通常具有复杂的几何形状、非线性材料行为以及多样化的荷载情况,导致经典的解析方法难以直接应用于计算和分析。
有限元法通过将复杂的结构问题离散化为许多小的有限元单元,并在每个单元上建立近似的数学模型,将复杂的问题转化为求解一系列简化的局部问题。这种离散化的方法使得有限元法能够有效地处理复杂的结构造型、各向异性材料特性以及不规则荷载。
通过有限元法,可以灵活地划分网格并调整网格密度,以适应不同区域的复杂性。
有限元法提供了一个强大的工具,可以对结构进行力学分析、优化设计以及可靠性评估。它已广泛应用于建筑、桥梁、飞机、汽车、石油和天然气管道等工程领域,为工程师和设计师提供了一种可靠、高效的方法来解决实际工程中的复杂问题。
有限元法作为一种广泛应用于解决复杂工程问题的数值方法,也仍存在网格划分复杂,网格的划分会极大的影响数值解的准确性和计算效率,在某些情况下,有限元分析对病态的元素会导致数值不稳定的现象,使得解失去可靠性和准确性,
不同的网格划分会导致不同的解存在着网格依赖性。并且基于单元插值的有限元法通常只有$C^0$连续,难以构造整体协调的$C^1$单元,无法直接处理板壳、相场等一系列高阶问题。\par
针对有限元法在解决实际工程中存在的问题,基于节点近似的无网格法得到日益广泛的关注。无网格法是一类根据离散节点位置信息直接建立形函数的方法,其形函数具有高阶光滑的特点,适用于薄板等高阶问题。
形函数构造过程中不依赖网格信息,适用于复杂区域的离散,能有效减轻网格畸变所引起的计算精度下降的问题。
\section{伽辽金无网格法研究历史及现状}
光滑粒子流体动力学(Smoothed Particle Hydrodynamics, SPH)是一种基于配点型无网格法的数值计算方法,用于模拟流体力学问题。它最早由Lucy(1977年)和Gingold与Monaghan(1977年)提出,并在之后的发展中得到了广泛的应用。
SPH方法采用核函数来对求解域进行离散化,将问题转化为一组粒子的运动和相互作用。与传统的网格方法不同,SPH方法不需要显式的网格结构,使得它可以处理复杂的流体流动和自由表面问题。
然而,早期的配点型无网格法在核函数的离散形式上并未满足一致性条件,这导致了计算精度和稳定性方面的存在着问题。\par
伽辽金无网格法源于1992年Nayroles等人提出的散射近似和散射元法(DEM, Diffuse Element Method)。
散射元法首先通过在离散点周围定义权重函数来对连续函数进行插值近似。然后利用伽辽金方法的思想,在离散点附近构建测试函数空间和试探函数空间。通过将加权最小二乘近似应用于离散方程的弱形式,得到一个离散方程系统,进而求解出问题的解。\par
1994年,Belytschko等人指出了散射近似(Scattered Data Approximation)实质上等同于移动最小二乘近似(Moving Least Squares, MLS)。
Belytschko等人提出准确求导的方法,而不是使用近似的散射导数。通过对移动最小二乘近似函数进行精确求导,引入背景网格和高阶高斯积分方法改善数值积分的精度,通过引入拉格朗日乘子,可以在伽辽金无单元法中有效地施加边界条件,从而提高计算精度和稳定性。在这些改进的基础上,Belytschko等人将方法命名为伽辽金无单元法(Element Free Galerkin Method, EFG)。\par
1995年,Liu等人在光滑粒子流体动力学(Smoothed Particle Hydrodynamics, SPH)核近似方法的基础上进行了改进。他们引入了核近似的多项式再生条件和校正函数,提出了再生核近似(Reproducing Kernel Approximation, RK)和再生核质点法(Reproducing Kernel Particle Method, RKPM)。
再生核近似是一种基于核函数的插值方法。在SPH中,原始的核函数仅是一种近似,可能无法满足一些重要的数学性质,如多项式再生条件。多项式再生条件要求核函数能够精确再现某些特定的多项式函数。为了满足这一条件,Liu等人引入了再生核函数,它是一种特殊的核函数,能够精确再现一定阶数的多项式函数。
再生核质点法(RKPM)是基于再生核近似的一种改进的SPH方法。在RKPM中,粒子的属性(如密度、速度)通过核函数进行插值近似。通过引入再生核函数和校正函数,可以更准确地重构粒子属性的离散化表示。再生核函数能够提供更好的逼近性能,而校正函数则用于修正核函数的近似误差。\par
伽辽金无单元法(EFG)和再生核质点法(RKPM)的提出标志着无网格法在数值计算领域的快速发展。随后,许多学者提出了各种特色各异的无网格法用于解决各式各样的问题。
例如:有限点法、自然单元法、物质点法、边界点法、最大熵近似无网格法、广义无网格法、奇异边界法、再生核有限元法、边界面法、准凸无网格法、径向点插值法、加权最小二乘无网格法、再生核径向基函数配点法、再生核近场动力学无网格法等
无网格法由于其不依赖于预定义的网格单元,因此在处理强非线性和大变形等复杂问题时被广泛应用。一些典型的应用包括裂纹扩展问题、金属冲压成形、橡胶材料大变形问题和薄壳结构大变形问题等。\par
无网格法由于形函数及其梯度通常为有理式,在采用伽辽金法进行求解的时候,传统基于多项式完备性建立的高斯积分法无法进行准确数值积分过程,导致伽辽金无网格法无法准确求解与其基函数阶次相同的解析解,即不满足变分一致性。
为了解决该问题,建立具有变分一致性的伽辽金无网格法成为无网格研究领域的热门问题。目前具有变分一致型的伽辽金无网格法主要是通过假定应变理论构造匹配的光滑梯度,光滑梯度为低阶多项式,采用低阶高斯积分法既能准确进行数值积分,保证计算误差的收敛阶次。同时,光滑梯度的构造过程仅需要计算传统无网格形函数,避免复杂耗时的形函数梯度计算,提高传统伽辽金无网格法的计算效率。 
\section{本文选题背景}
变分一致型伽辽金无网格法可追溯至2004年Chen等人提出的稳定节点积分法,该方法从伽辽金法的线性准确性条件出发,提出了线性积分条件。伽辽金数值积分方需要满足积分约束条件,才能求解线性问题,即满足线性的变分一致性。该方法假设光滑梯度为常数,通过满足线性积分约束条件建立光滑梯度。
段庆林等人将线性积分约束条件推广至高阶情况,并通过高阶积分约束条件计算高斯积分点处光滑梯度值,使得传统高斯积分法满足变分一致性,并称该方法为一致型无单元伽辽金法。
王东东和吴俊超提出了再生光滑理论框架[15],该框架具有与传统无网格形函数相类似的表达式,统一了以假定应变为基础的变分一致型光滑梯度构造方案。在该理论框架下,光滑梯度构造过程可重新合理优化数值积分采样点位置和权重,从而减少无网格形函数的计算量,提高计算效率。
\par
除了假定应变为基础的变分一致型无网格法外,Chen等人[16]提出修正变分积分法,该方法在伽辽金弱形式中采用不同形式权函数和试函数以满足高阶变分一致性,但该方法会丧失刚度矩阵的对称性;王东东和吴俊超[17]利用两层次积分域得到的刚度矩阵进行合理组合,消除二阶误差项以满足二次变分一致性,但该方法难以推广至三维及高阶情况。\par
为了满足全域的变分一致性,满足积分约束条件的无网格数值积分方案需要配合具有变分一致性的本质边界条件施加方案,传统无网格形函数在自身节点处不具有插值性,
Belytschko等人[21]最早采用拉格朗日乘子法施加本质边界条件,该方法需要引入额外自由度离散拉格朗日乘子,当采用变分一致型无网格数值积分方案时,拉格朗日乘子的自由度需要和光滑梯度构造过程中积分点的位置保持一致,以满足变分一致性。采用过多自由度离散拉格朗日乘子将导致整体刚度矩阵出现奇异,以致于该方法不适合高阶的变分一致型伽辽金无网格法。
罚函数法[22]施加本质边界条件无需额外增加自由度,数值实现简单。广泛应用于伽辽金无网格法。但该方法的计算误差依赖于人工经验参数,且不具有变分一致性,不能保证计算精度。
Nitshce法[23]是目前变分一致型无网格法主要采用的本质边界条件施加方法,该方法在修正变分原理的基础上引入罚函数法作为保证刚度矩阵的正定性。但是在积分一致的数值积分方案中已经不需要的无网格高阶梯度被重新引入,降低了无网格分析的计算效率。
同时,Nitsche法中的稳定性还是需要人工经验参数,过大或过小的人工参数都将导致计算精度的降低。
\par
另一类无网格施加本质边界条件的方法是试图恢复无网格形函数的插值性。
Fernández-Méndez与Huerta[23]通过修改无网格形函数构造过程中核函数的权重,使得无网格形函数在边界处具有插值性,但该方法无法满足积分约束条件。Hillman和Lin[25]在修正变分法中引入具有插值性无网格近似,但该方法改变了解的空间。
Chen等人[16]采用转换矩阵,将无网格法中的节点系数重新与物理值建立联系,从而直接施加本质边界条件,并称该方法为变换法。
王东东[24]将变换法引入稳定节点积分法中,并对其进行了修正,以保证数值积分的一致性。然而,变换法中转换矩阵需作用于整体刚度矩阵,计算量大,不适用于大规模计算。
\section{本文主要内容}

本文研究将发展基于Hellinger-Reissner原理的变分一致型伽辽金无网格法,依托于Hellinger-Reissner原理内嵌本质边界条件的特点,通过对Hellinger-Reissner变分原理弱形式中的变量采用再生核近似与在每个背景积分域中假设为多项式进行混合离散的形式,
建立具有变分一致性且不依赖于人工经验参数的本质边界条件施加方法,提高计算精度,用再生光滑梯度替换传统无网格形函数梯度,减少无网格形函数的计算量,有效提高计算效率。
具体内容如下:\par
(1)基于弹性力学问题,研究对位移和应力采用混合离散的Hellinger-Reissner原理的变分一致型伽辽金无网格法,推导出具有变分一致性的无网格本质边界条件施加方案,通过分片实验和典型弹性力学问题进一步验证该方法的变分一致性,并详细对比所提方法与传统本质边界条件施加方案之间的区别,验证所提方法的计算精度和效率。\par
(2)基于高阶薄板问题,研究对挠度和弯矩采用混合离散的Hellinger-Reissner原理的变分一致型伽辽金无网格法,推导出具有变分一致性的无网格本质边界条件施加方案,通过分片实验和薄板问题进一步验证该方法的变分一致性,并详细对比所提方法与传统本质边界条件施加方案之间的区别,验证所提方法的计算精度和效率。\par
(3)基于薄板动力分析问题,\par




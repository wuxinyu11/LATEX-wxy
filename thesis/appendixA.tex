\chapter{弹性力学问题HR变分原理的本质边界条件施加方法推导过程}\label{A}
% \addcontentsline{toc}{chapter}{附录A:弹性力学问题HR变分原理的本质边界条件施加方法推导过程}
通过引入式(\ref{g1})、(\ref{case1})对式(\ref{inference})中的第一项进行推导得到:
\begin{equation}\label{CH4-K}
\begin{split}
    &\sum_{C=1}^{N\!C}\tilde{\pmb g}^T_{jI}C_{ijkl}\pmb{G}^{-1}\tilde{\pmb g}_{kJ}\\
&=\sum_{C=1}^{N\!C}\tilde{\pmb g}^T_{jI}C_{ijkl}\pmb{G}^{-1}\pmb{G}\pmb{G}^{-1}\tilde{\pmb g}_{kJ}\\
&=\sum_{C=1}^{N\!C}\int_{\Omega_C}\tilde{\pmb g}_{jI}^T\pmb{G}^{-1}\pmb{p}^{[p-1]}(\pmb x)C_{ijkl}\pmb{p}^{[p-1]T}(\pmb{x})\pmb{G}^{-1}\tilde{\pmb g}_{kJ}d\Omega\\
&=\sum_{C=1}^{N\!C}\int_{\Omega_C}\tilde{\Psi}_{I,j}C_{ijkl}\tilde{\Psi}_{I,k}d\Omega\\
&=\sum_{C=1}^{N\!C}\int_{\Omega_C}\delta\tilde{\varepsilon}_{ij}^hC_{ijkl}\tilde{\varepsilon}_{kl}^hd\Omega\\
&=\int_{\Omega}\tilde{\pmb{B}}_I^T\pmb{D}\tilde{\pmb{B}}_Jd\Omega\\
&=\pmb{K}
\end{split}
\end{equation}
\newpage
通过引入式(\ref{g3})、(\ref{case1})、(\ref{CH4-ui})对式(\ref{inference})中的第二、三项进行推导得到:
\begin{equation}\label{CH4-tildeK}
\begin{split}
    &-\sum_{C=1}^{N\!C}C_{ijkl}(\bar{\pmb g}_{jI}^T\pmb{G}^{-1}\tilde{\pmb g}_{kJ}+\tilde{\pmb g}_{jI}^T\pmb{G}^{-1}\bar{\pmb g}_{kJ})\\
    &=-\sum_{C=1}^{N\!C}(\int_{\Gamma^g\cap\partial\Omega_C}\Psi_In_jC_{ijkl}\pmb{p}^{[p-1]}\pmb{G}^{-1}\tilde{\pmb g}_{kJ}d\Gamma
    +\int_{\Gamma^g\cap\partial\Omega_C}C_{ijkl}\pmb{p}^{[p-1]}\pmb{G}^{-1}\tilde{\pmb g}_{jI}n_k\Psi_Jd\Gamma)\\
    &=-\sum_{C=1}^{N\!C}(\int_{\Gamma^g\cap\partial\Omega_C}\Psi_In_jC_{ijkl}\tilde{\Psi}_{J,k}d\Gamma
    +\int_{\Gamma^g\cap\partial\Omega_C}C_{ijkl}\sum_{J=1}^{N\!P}\tilde{\Psi}_{I,j}\delta d_{iI}n_k\sum_{I=1}^{N\!P}\Psi_Jd\Gamma)\\
    &=-\sum_{C=1}^{N\!C}(\int_{\Gamma^g\cap\partial\Omega}\delta u_i^hn_jC_{ijkl}\tilde{\varepsilon}_{kl}^hd\Gamma+\int_{\Gamma^g\cap\partial\Omega_C}\delta\tilde{\varepsilon}_{ij}^hC_{ijkl}n_ku^h_ld\Gamma)\\
    &=-(\int_{\Gamma^g}\Psi_I\pmb{N}\pmb{D}\tilde{\pmb{B}}_Jd\Gamma+\int_{\Gamma^g}\tilde{\pmb{B}}_I^T\pmb{D}\pmb{N}^T\Psi_Jd\Gamma)\\
    &=\tilde{\pmb{K}}
\end{split}
\end{equation}
通过引入式(\ref{g3})、(\ref{case2})对式(\ref{inference})中的第四项进行推导得到:
\begin{equation}\label{CH4-barK}
\begin{split}
    &\sum_{C=1}^{N\!C}C_{ijkl}\bar{\pmb g}^T_{jI}\pmb{G}^{-1}\bar{\pmb g}_{kJ}\\
    &=\sum_{C=1}^{N\!C}\int_{\Gamma^g\cap\partial\Omega_C}C_{ijkl}\bar{\pmb g}_{jI}^T\pmb{G}^{-1}\pmb{p}^{[p-1]}\Psi_Jn_kd\Gamma\\
    &=\sum_{C=1}^{N\!C}\int_{\Gamma^g\cap\partial\Omega_C}C_{ijkl}\bar{\Psi}_{I,j}n_k\Psi_{J}d\Gamma\\
    &=\sum_{C=1}^{N\!C}\int_{\Gamma^g\cap\partial\Omega_C}C_{ijkl}\delta\bar{\varepsilon}_{ij}^hn_ku_ld\Gamma\\
    &=\delta\pmb{d}_I^T\int_{\Gamma^g\cap\partial\Omega_C}\bar{\pmb{B}}_I^T\pmb{D}\pmb{N}^T\Psi_Jd\Gamma\\
    &=\bar{\pmb{K}}
\end{split}
\end{equation}
\newpage
通过引入式(\ref{g4})、(\ref{case1})对式(\ref{inference})中的第五项进行推导得到:
\begin{equation}\label{CH4-tildef}
\begin{split}
    &-\sum_{C=1}^{N\!C}C_{ijkl}\tilde{\pmb g}^T_{jI}\pmb{G}^{-1}\hat{\pmb g}_{kl}\\
    &=-\sum_{C=1}^{N\!C}C_{ijkl}\tilde{\pmb g}^T_{jI}\pmb{G}^{-1}\int_{\Gamma^g\cap\partial\Omega_C}\pmb{p}^{[p-1]}n_lg_kd\Gamma\\
    &=-\sum_{C=1}^{N\!C}\int_{\Gamma^g\cap\partial\Omega_C}C_{ijkl}\tilde{\Psi}_{I,j}n_lg_kd\Gamma\\
    &=-\sum_{C=1}^{N\!C}\int_{\Gamma^g\cap\partial\Omega_C}C_{ijkl}\delta\tilde{\varepsilon}_{ij}^hn_lg_kd\Gamma\\
    &=-\int_{\Gamma^g\cap\partial\Omega_C}\tilde{\pmb{B}}_I^T\pmb{D}\pmb{N}^T\pmb{g}d\Gamma\\
    &=\tilde{\pmb f}
\end{split}
\end{equation}
通过引入式(\ref{g4})、(\ref{case2})对式(\ref{inference})中的第六项进行推导得到:
\begin{equation}\label{CH4-barf}
\begin{split}
    &\sum_{C=1}^{N\!C}C_{ijkl}\bar{\pmb g}^T_{jI}\pmb{G}^{-1}\hat{\pmb g}_{kl}\\
    &=\sum_{C=1}^{N\!C}C_{ijkl}\bar{\pmb g}^T_{jI}\pmb{G}^{-1}\int_{\Gamma^g\cap\partial\Omega_C}\pmb{p}^{[p-1]}n_lg_kd\Gamma\\
    &=\sum_{C=1}^{N\!C}\int_{\Gamma^g\cap\partial\Omega_C}C_{ijkl}\bar{\Psi}_{I,j}n_lg_kd\Gamma\\
    &=\sum_{C=1}^{N\!C}\int_{\Gamma^g\cap\partial\Omega_C}C_{ijkl}\delta\bar{\varepsilon}_{ij}^hn_lg_kd\Gamma\\
    &=\int_{\Gamma^g\cap\partial\Omega_C}\bar{\pmb{B}}_I^T\pmb{D}\pmb{N}^T\pmb{g}d\Gamma\\
    &=\bar{\pmb f}
\end{split}
\end{equation}


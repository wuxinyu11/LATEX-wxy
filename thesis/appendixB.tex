\chapter{薄板问题HR变分原理的本质边界条件施加方法推导过程}\label{B}
在本附录中详细推导了第四章方程式(\ref{Pzuzhuang})中的刚度矩阵和力向量的推导过程,该等式为:
\begin{equation}
    \begin{split}
        &-\sum_{C=1}^{N\!C}(\tilde{\pmb g}_{\alpha\beta I}^T-\bar{\pmb g}_{\alpha\beta I}^T)\pmb a_{\alpha\beta}\\
        &=\sum_{C=1}^{N\!C}(\tilde{\pmb g}_{\alpha\beta I}^T-\bar{\pmb g}_{\alpha\beta I}^T)D_{\alpha\beta\gamma\eta}\pmb{G}^{-1}(\sum_{J=1}^{N\!P}(\tilde{\pmb g}_{\gamma\eta J}-\bar{\pmb g}_{\gamma\eta J})d_I+\hat{\pmb g}_{\gamma\eta})\\
        &=\sum_{C=1}^{N\!C}
        \left(\begin{split}
        &\sum_{J=1}^{N\!P}\underbrace{D_{\alpha\beta\gamma\eta}\tilde{\pmb g}_{\alpha\beta I}^T\pmb G^{-1}\tilde{\pmb g}_{\gamma\eta J}}_{\pmb{K}}d_J\\
        &+\sum_{J=1}^{N\!P}\underbrace{D_{\alpha\beta\gamma\eta}(-\bar{\pmb g}_{\alpha\beta I}^T\pmb G^{-1}\tilde{\pmb g}_{\gamma\eta J}-\tilde{\pmb g}_{\alpha\beta I}^T\pmb G^{-1}\tilde{\pmb g}_{\gamma\eta J})}_{\tilde{\pmb K}}d_J\\
        &+\sum_{J=1}^{N\!P}\underbrace{D_{\alpha\beta\gamma\eta}\bar{\pmb g}_{\alpha\beta I}^T\pmb G^{-1}\tilde{\pmb g}_{\gamma\eta J}}_{\bar{\pmb K}}d_J\\
        &-\underbrace{(-D_{\alpha\beta\gamma\eta}\tilde{\pmb g}_{\alpha\beta I}^T\pmb G^{-1}\hat{\pmb g}_{\gamma\eta })}_{\tilde{\pmb f}}\\
        &-\underbrace{D_{\alpha\beta\gamma\eta}\bar{\pmb g}_{\alpha\beta I}^T\pmb G^{-1}\hat{\pmb g}_{\gamma\eta }}_{\bar{\pmb f}}\\
        \end{split}\right)\\
        &=\sum_{J=1}^{N\!P}(\pmb{K}+\tilde{\pmb{K}}+\bar{\pmb{K}})\pmb d_J-\tilde{\pmb f}-\bar{\pmb f}
\end{split}
\end{equation}
\newpage
式(\ref{PHR11})中的常规刚度矩阵$\pmb{K}$:
% 通过引入式(\ref{PG})、(\ref{PTPSI})对式(\ref{Pzuzhuang})中的第一项进行推导可以得到:
\begin{equation}
\begin{split}
    &\sum_{C=1}^{N\!C}D_{\alpha\beta\gamma\eta}\tilde{\pmb g}_{\alpha\beta I}^T\pmb G^{-1}\tilde{\pmb g}_{\gamma\eta J}\\
    &=\sum_{C=1}^{N\!C}D_{\alpha\beta\gamma\eta}\tilde{\pmb g}_{\alpha\beta I}^T\pmb G^{-1}\int_{\Omega_C}\pmb{p}^{[p-2]}\pmb{p}^{[p-2]T}d\Omega\pmb G^{-1}\tilde{\pmb g}_{\gamma\eta J}\\
    &=\sum_{C=1}^{N\!C}D_{\alpha\beta\gamma\eta}\int_{\Omega_C}\tilde{\pmb g}_{\alpha\beta I}^T\pmb G^{-1}\pmb{p}^{[p-2]}\pmb{p}^{[p-2]T}\pmb G^{-1}\tilde{\pmb g}_{\gamma\eta J}d\Omega \\
    &=\int_{\Omega}\tilde{\Psi}_{I,\alpha\beta}D_{\alpha\beta\gamma\eta}\tilde{\Psi}_{J,\gamma\eta}d\Omega \\
    &=\pmb{K}
\end{split}
\end{equation}
\newpage
式(\ref{PHR21})中的一致性刚度矩阵$\tilde{\pmb{K}}$:
% 通过引入式(\ref{PTPSI})、(\ref{MVP1})对式(\ref{Pzuzhuang})中的第二项进行推导可以得到:
\begin{equation}
\begin{split}
    &-\sum_{C=1}^{N\!C}D_{\alpha\beta\gamma\eta}(\bar{\pmb g}_{\alpha\beta I}^T\pmb G^{-1}\tilde{\pmb g}_{\gamma\eta J}+\tilde{\pmb g}_{\alpha\beta I}^T\pmb G^{-1}\tilde{\pmb g}_{\gamma\eta J})\\
    &=\sum_{C=1}^{N\!C}D_{\alpha\beta\gamma\eta}\int_{{\Gamma_w}\cap{\Gamma_C}}\Psi_I(n_{\alpha}
    \pmb{p}^{[p-2]T}_{,\beta}\pmb G^{-1}\tilde{\pmb g}_{\gamma\eta J}+\pmb{p}^{[p-2]T}_{,\xi}\pmb G^{-1}\tilde{\pmb g}_{\gamma\eta J}s_{\alpha}n_{\beta}s_{\xi})d\Gamma\\
    &+\sum_{C=1}^{N\!C}D_{\alpha\beta\gamma\eta}\int_{{\Gamma_w}\cap{\Gamma_C}}(n_{\gamma}
    \tilde{\pmb g}_{\alpha\beta I}^T\pmb G^{-1}\pmb{p}^{[p-2]}_{,\eta}+\tilde{\pmb g}_{\alpha\beta I}^T
    \pmb G^{-1}\pmb{p}^{[p-2]}_{,\xi}s_{\gamma}n_{\eta}s_{\xi})\Psi_Jd\Gamma\\
    &-\sum_{C=1}^{N\!C}D_{\alpha\beta\gamma\eta}\int_{{\Gamma_{\theta}}\cap{\Gamma_C}}\Psi_{I,n}
    \pmb{p}^{[p-2]T}\pmb G^{-1}\tilde{\pmb g}_{\gamma\eta J}n_{\alpha}n_{\beta}d\Gamma
    -\sum_{C=1}^{N\!C}D_{\alpha\beta\gamma\eta}\int_{{\Gamma_{\theta}}\cap{\Gamma_C}}
    \tilde{\pmb g}_{\alpha\beta I}^T\pmb G^{-1}\pmb{p}^{[p-2]}n_{\gamma}n_{\eta}\Psi_{J,n}d\Gamma\\
    &-\sum_{C=1}^{N\!C}D_{\alpha\beta\gamma\eta}[[\Psi_I\pmb{p}^{[p-2]T}\pmb G^{-1}\tilde{\pmb g}_{\gamma\eta J}n_{\alpha}s_{\beta}]]_{x\in{c_w}\cap{c_C}}
    -\sum_{C=1}^{N\!C}D_{\alpha\beta\gamma\eta}[[\tilde{\pmb g}_{\alpha\beta I}^T\pmb G^{-1}\pmb{p}^{[p-2]}n_{\gamma}s_{\eta}\Psi_J]]_{x\in{c_w}\cap{c_C}}\\
    &=-\sum_{C=1}^{N\!C}\int_{{\Gamma_w}\cap{\Gamma_C}}\Psi_I(-D_{\alpha\beta\gamma\eta}(n_{\alpha}\frac{\partial}{\partial x_{\beta}}+s_{\alpha}n_{\beta}s_{\xi}\frac{\partial}{\partial x_{\xi}}))\tilde{\Psi}_{J,\gamma\eta}d\Gamma\\
    &-\sum_{C=1}^{N\!C}\int_{{\Gamma_w}\cap{\Gamma_C}}(-D_{\alpha\beta\gamma\eta}(n_{\gamma}\frac{\partial}{\partial x_{\eta}}+s_{\gamma}n_{\eta}s_{\xi}\frac{\partial}{\partial x_{\xi}}))\tilde{\Psi}_{I,\alpha\beta}\Psi_Jd\Gamma\\
    &+\sum_{C=1}^{N\!C}\int_{{\Gamma_{\theta}}\cap{\Gamma_C}}\Psi_{I,n}(-D_{\alpha\beta\gamma\eta}n_{\alpha}n_{\beta})\tilde{\Psi}_{J,\gamma\eta}d\Gamma
    +\sum_{C=1}^{N\!C}\int_{{\Gamma_{\theta}}\cap{\Gamma_C}}(-D_{\alpha\beta\gamma\eta}n_{\gamma}n_{\eta})\tilde{\Psi}_{I,\alpha\beta}\Psi_{J,n}d\Gamma\\
    &+\sum_{C=1}^{N\!C}[[\Psi_I(-D_{\alpha\beta\gamma\eta}n_{\gamma}s_{\eta})\tilde{\Psi}_{J,\gamma\eta}]]_{x\in{c_w}\cap{c_C}}+\sum_{C=1}^{N\!C}[[(-D_{\alpha\beta\gamma\eta}n_{\gamma}s_{\eta})\tilde{\Psi}_{I,\alpha\beta}\Psi_J]]_{x\in{c_w}\cap{c_C}}\\
    &=-\int_{\Gamma_w}\Psi_I\mathcal{V}_{\alpha\beta}\tilde{\Psi}_{J,\alpha\beta}d\Gamma+\int_{\Gamma_{\theta}}\Psi_{I,n}\mathcal{M}_{\alpha\beta}\tilde{\Psi}_{J,\alpha\beta}d\Gamma+[[\Psi_I\mathcal{P}_{\alpha\beta}\tilde{\Psi}_{J,\alpha\beta}]]_{x\in{c_w}}\\
    &-\int_{\Gamma_w}\mathcal{V}_{\alpha\beta}\tilde{\Psi}_{I,\alpha\beta}\Psi_Jd\Gamma+\int_{\Gamma_{\theta}}\mathcal{M}_{\alpha\beta}\tilde{\Psi}_{I,\alpha\beta}\Psi_{J,n}d\Gamma+[[\mathcal{P}_{\alpha\beta}\tilde{\Psi}_{I,\alpha\beta}\Psi_J]]_{x\in{c_w}}\\
    &=\tilde{\pmb K}
\end{split}
\end{equation}
\newpage
式(\ref{PHR31})中的稳定刚度矩阵$\bar{\pmb{K}}$:
% 通过引入式(\ref{PBPSI})、(\ref{MVP1})对式(\ref{Pzuzhuang})中的第三项进行推导可以得到:
\begin{equation}
\begin{split}
    &\sum_{C=1}^{N\!C}D_{\alpha\beta\gamma\eta}\bar{\pmb g}_{\alpha\beta I}^T\pmb G^{-1}\tilde{\pmb g}_{\gamma\eta J}\\
    &=\sum_{C=1}^{N\!C}D_{\alpha\beta\gamma\eta}\int_{{\Gamma_w}\cap{\Gamma_C}}(n_{\gamma}
    \bar{\pmb g}_{\alpha\beta I}^T\pmb G^{-1}\pmb{p}^{[p-2]}_{,\eta}+\tilde{\pmb g}_{\alpha\beta I}^T
    \pmb G^{-1}\pmb{p}^{[p-2]}_{,\xi}s_{\gamma}n_{\eta}s_{\xi})\Psi_Jd\Gamma\\
    &-\sum_{C=1}^{N\!C}D_{\alpha\beta\gamma\eta}\int_{{\Gamma_{\theta}}\cap{\Gamma_C}}
    \bar{\pmb g}_{\alpha\beta I}^T\pmb G^{-1}\pmb{p}^{[p-2]}n_{\gamma}n_{
    \eta}\Psi_{J,n}d\Gamma
    -\sum_{C=1}^{N\!C}D_{\alpha\beta\gamma\eta}[[\bar{\pmb g}_{\alpha\beta I}^T\pmb G^{-1}\pmb{p}^{[p-2]}n_{\gamma}s_{\eta}\Psi_J]]_{x\in{c_w}\cap{c_C}}\\
    &=\sum_{C=1}^{N\!C}\int_{{\Gamma_w}\cap{\Gamma_C}}(-D_{\alpha\beta\gamma\eta}(n_{\gamma}\frac{\partial}{\partial x_{\eta}}+s_{\gamma}n_{\eta}s_{\xi}\frac{\partial}{\partial x_{\xi}}))\bar{\Psi}_{I,\alpha\beta}\Psi_Jd\Gamma\\
    &-\sum_{C=1}^{N\!C}\int_{{\Gamma_{\theta}}\cap{\Gamma_C}}(-D_{\alpha\beta\gamma\eta}n_{\gamma}n_{\eta})\bar{\Psi}_{I,\alpha\beta}\Psi_{J,n}d\Gamma
    -[[(-D_{\alpha\beta\gamma\eta}n_{\gamma}s_{\eta})\bar{\Psi}_{I,\alpha\beta}\Psi_J]]_{x\in{c_w}\cap{c_C}}\\
    &=\int_{\Gamma_w}\mathcal{V}_{\alpha\beta}\bar{\Psi}_{I,\alpha\beta}\Psi_Jd\Gamma-\int_{\Gamma_{\theta}}\mathcal{M}_{\alpha\beta}\bar{\Psi}_{I,\alpha\beta}\Psi_{J,n}d\Gamma+[[\mathcal{P}_{\alpha\beta}\bar{\Psi}_{I,\alpha\beta}\Psi_J]]_{x\in{c_w}}\\
    &=\bar{\pmb K}
\end{split}
\end{equation}
\par
式(\ref{PHR22})中的一致性力向量$\tilde{\pmb{f}}$:
% 通过引入式(\ref{PTPSI})、(\ref{MVP1})对式(\ref{Pzuzhuang})中的第四项进行推导可以得到:
\begin{equation}
\begin{split}
    &-\sum_{C=1}^{N\!C}D_{\alpha\beta\gamma\eta}\tilde{\pmb g}^T_{\alpha\beta I}\pmb G^{-1}\hat{\pmb g}_{\gamma\eta}\\
    &=\sum_{C=1}^{N\!C}D_{\alpha\beta\gamma\eta}\int_{{\Gamma_w}\cap{\Gamma_C}}(n_{\gamma}
    \tilde{\pmb g}_{\alpha\beta I}^T\pmb G^{-1}\pmb{p}^{[p-2]}_{,\eta}+\tilde{\pmb g}_{\alpha\beta I}^T
    \pmb G^{-1}\pmb{p}^{[p-2]}_{,\xi}s_{\gamma}n_{\eta}s_{\xi})\bar{w}d\Gamma\\
    &-\sum_{C=1}^{N\!C}D_{\alpha\beta\gamma\eta}\int_{{\Gamma_{\theta}}\cap{\Gamma_{C}}}\tilde{\pmb g}^T_{\alpha\beta I}\pmb G^{-1}\pmb{p}^{[p-2]}n_{\gamma}n_{\eta}\bar{\theta}_nd\Gamma-[[\tilde{\pmb g}_{\alpha\beta I}\pmb G^{-1}\pmb{p}^{[p-2]}n_{\gamma}s_{\eta}\bar{w}]]_{x\in{c_w}\cap{c_C}}\\
    &=-\sum_{C=1}^{N\!C}\int_{{\Gamma_w}\cap{\Gamma_C}}(-D_{\alpha\beta\gamma\eta}(n_{\gamma}\frac{\partial}{\partial x_{\eta}}+s_{\gamma}n_{\eta}s_{\xi}\frac{\partial}{\partial x_{\xi}}))\tilde{\Psi}_{I,\alpha\beta}\bar{w}d\Gamma\\
    &+\sum_{C=1}^{N\!C}\int_{{\Gamma_{\theta}}\cap{\Gamma_C}}(-D_{\alpha\beta\gamma\eta}n_{\gamma}n_{\eta})\tilde{\Psi}_{I,\alpha\beta}\bar{\theta}_{\pmb n}d\Gamma-[[(-D_{\alpha\beta\gamma\eta}n_{\gamma}s_{\eta}\tilde{\Psi}_{I,\alpha\beta}\bar{w})]]_{x\in{c_w}\cap{c_C}}\\
    &=\int_{\Gamma_w}\mathcal{V}_{\alpha\beta}\tilde{\Psi}_{I,\alpha\beta}\bar{w}d\Gamma+\int_{\Gamma_{\theta}}\mathcal{M}_{\alpha\beta}\tilde{\Psi}_{I,\alpha\beta}\bar{\theta}_{\pmb n}d\Gamma+[[\mathcal{P}_{\alpha\beta}\tilde{\Psi}_{I,\alpha\beta}\bar{w}]]_{x\in{c_w}}\\
    &=\tilde{\pmb f}
\end{split}
\end{equation}
\newpage
式(\ref{PHR32})中的稳定力向量$\bar{\pmb{f}}$:
% 通过引入式(\ref{PBPSI})、(\ref{MVP1})对式(\ref{Pzuzhuang})中的第五项进行推导可以得到:
\begin{equation}
 \begin{split}
    &-\sum_{C=1}^{N\!C}D_{\alpha\beta\gamma\eta}\bar{\pmb g}_{\alpha\beta I}^T\pmb G^{-1}\hat{\pmb g}_{\gamma\eta }\\
    &=\sum_{C=1}^{N\!C}D_{\alpha\beta\gamma\eta}\int_{{\Gamma_w}\cap{\Gamma_C}}(n_{\eta}
    \bar{\pmb g}_{\alpha\beta I}^T\pmb G^{-1}\pmb{p}^{[p-2]}_{,\gamma}+\bar{\pmb g}_{\alpha\beta I}^T
   \pmb G^{-1}\pmb{p}^{[p-2]}_{,\xi}s_{\gamma}n_{\eta}s_{\xi})\bar{w}d\Gamma\\
    &-\sum_{C=1}^{N\!C}D_{\alpha\beta\gamma\eta}\int_{{\Gamma_{\theta}}\cap{\Gamma_{C}}}\bar{\pmb g}^T_{\alpha\beta I}\pmb G^{-1}\pmb{p}^{[p-2]}n_{\gamma}n_{\eta}\bar{\theta}_nd\Gamma-[[\bar{\pmb g}_{\alpha\beta I}\pmb G^{-1}\pmb{p}^{[p-2]}n_{\gamma}s_{\eta}\bar{w}]]_{x\in{c_w}\cap{c_C}}\\
    &=\sum_{C=1}^{N\!C}\int_{{\Gamma_w}\cap{\Gamma_C}}(-D_{\alpha\beta\gamma\eta}(n_{\gamma}\frac{\partial}{\partial x_{\eta}}+s_{\gamma}n_{\eta}s_{\xi}\frac{\partial}{\partial x_{\xi}}))\bar{\Psi}_{I,\alpha\beta}\bar{w}d\Gamma\\
    &-\sum_{C=1}^{N\!C}\int_{{\Gamma_{\theta}}\cap{\Gamma_C}}(-D_{\alpha\beta\gamma\eta}n_{\gamma}n_{\eta})\bar{\Psi}_{I,\alpha\beta}\bar{\theta}_{\pmb n}d\Gamma-[[(-D_{\alpha\beta\gamma\eta}n_{\gamma}s_{\eta}\bar{\Psi}_{I,\alpha\beta}\bar{w})]]_{x\in{c_w}\cap{c_C}}\\
    &=\int_{\Gamma_w}\mathcal{V}_{\alpha\beta}\bar{\Psi}_{I,\alpha\beta}\bar{w}d\Gamma+\int_{\Gamma_{\theta}}\mathcal{M}_{\alpha\beta}\bar{\Psi}_{I,\alpha\beta}\bar{\theta}_{\pmb n}d\Gamma+[[\mathcal{P}_{\alpha\beta}\bar{\Psi}_{I,\alpha\beta}\bar{w}]]_{x\in{c_w}}\\
    &=\bar{\pmb f}
\end{split}
\end{equation}

\documentclass[a4paper]{ctexbook}
\usepackage{amsmath,titlesec,amssymb,graphicx,floatrow}
\usepackage{xeCJK,hyperref}
\begin{document}
\chapter{无网格形函数}
\section{再生核无网格近似}
无网格法通过如图所示的问题域$\Omega$和边界$\Gamma$上布置一系列无网格节点$\{\pmb{x}_I\}^{N\!P}_{I=1}$进行离散,其中$N\!P$表示无网格节点数量。每个无网格节点$\pmb{x}_I$对应的形函数为$\Psi(\pmb{x})$,影响域为$supp(\pmb{x}_I)$,
每一个节点的影响域$supp(\pmb{x}_I)$满足$\Omega\in^{N\!P}_{I=1}supp(\pmb{x}_I)$。不失为一般性,考虑任意变量$u(\pmb{x})$,其对应的无网格近似函数$u^h(\pmb{x})$表示为:
\begin{equation}
\begin{split}
    u^h(\pmb{x})=\sum_{I=1}^{N\!P}\Psi_I(\pmb{x})d_I
\end{split}
\end{equation}
其中,$d_I$表示与节点$\pmb{x}_I$对应的系数\par
根据再生核近似理论[],无网格形函数可以假设为:
\begin{equation}
\begin{split}
    \Psi_I(\pmb{x})=\sum_{I=1}^{N\!P}\pmb{p}^T(\pmb{x}_I-\pmb{x})\pmb{c}(\pmb{x})\phi_s(\pmb{x}_I-\pmb{x})
\end{split}
\end{equation}
式中,$\pmb{p}(\pmb{x})$表示为$p$阶的多项式基函数向量,表达式为:
\begin{equation}
\begin{split}
    \pmb{p}(\pmb{x})=\{1,x,y,\dotsb,x^iy^i,\dotsb,y^p\}.0\le i+j \le p
\end{split}
\end{equation}
而$\phi_s(\pmb{x}_I-\pmb{x})$是附属于节点$\pmb{x}_I$的核函数,其影响域的大小由影响域尺寸$s$决定,核函数以及其影响域的大小共同决定了无网格形函数的局部紧支性和光滑性。对应二维问题,一般情况下核函数$\phi_s(\pmb{x}_I-\pmb{x})$的影响域为圆形域或者矩形域,可由下列公式进行得到:
\begin{equation}
\begin{split}
    \phi_s(\pmb{x}_I-\pmb{x})=\phi_{s_x}(r_x)\phi_{s_y}(r_y),r_x=\frac{\lvert x_I-x\rvert}{s_x},r_y=\frac{\lvert y_I-y \rvert}{s_y}
\end{split}
\end{equation}
其中$s_x$和$s_y$分别为$x$和$y$方向上影响域的大小,计算时一般使得两个方向上的影响域大小相等即$s_x=s_y=s$。选取核函数时一般遵循核函数阶次$m$大于等于基函数阶次$p(m\ge p)$的原则。针对二阶势问题的弹性力学问题,无网格基函数一般选择二阶或者三阶,而核函数$\phi_s(\pmb{x}_I-\pmb{x})$则选取三次样条函数:
\begin{equation}
\begin{split}
    \phi(r)=\frac{1}{3!}
\begin{cases}
    (2-2r)^3-4(1-2r)^3 &r\le \frac{1}{2}\\
    (2-2r)^3&\frac{1}{2}<r\le 1\\
    0&r>1
\end{cases}
\end{split}
\end{equation}
针对高阶薄板问题,无网格基函数一般选择三阶或者四阶,而核函数$\phi_s(\pmb{x}_I-\pmb{x})$则选择五次样条函数:
\begin{equation}
\begin{split}
    \phi(r)=\frac{1}{5!}
\begin{cases}
    (3-3r)^5-6(2-3r)^5+15(1-3r)^5&r\le\frac{1}{3}\\
    (3-3r)^5-6(2-3r)^5&\frac{1}{3}<r\le\frac{2}{3}\\
    (3-3r)^5&\frac{2}{3}<r\le1\\
    0&r>1
\end{cases}
\end{split}
\end{equation}\par
无网格形函数表达式(1.2)中的$\pmb{c}$为待定系数向量,该表达式可以通过满足再生条件确定:
\begin{equation}
\begin{split}
    \sum_{I=1}^{N\!P}\Psi_I(\pmb{x})\pmb{p}(\pmb{x}_I-\pmb{x})=\pmb{0}
\end{split}
\end{equation}
将无网格形函数表达式(1.2)代入再生条件(1.7)中,可以得到:
\begin{equation}
\begin{split}
    \pmb{c}(\pmb{x})=\pmb{A}^{-1}(\pmb{x})\pmb{p}(\pmb{0})
\end{split}
\end{equation}
其中$\pmb{A}(\pmb{x})$表示矩量矩阵,表达式为:
\begin{equation}
\begin{split}
    \pmb{A}(\pmb{x})=\sum_{I=1}^{N\!P}\pmb{p}(\pmb{x}_I-\pmb{x})\pmb{p}^T(\pmb{x}_I-\pmb{x})\phi_s(\pmb{x}_I-\pmb{x})
\end{split}
\end{equation}\par
将$\pmb{c}(\pmb{x})$代入到式子(1.2)中得到再生核无网格形函数的表达式:
\begin{equation}
\begin{split}
    \Psi_I(\pmb{x})=\pmb{p}^T(\pmb{0})\pmb{A}^{-1}(\pmb{x})\pmb{x}_I-\pmb{x}\phi_s(\pmb{x}_I-\pmb{x})
\end{split}
\end{equation}\par
无网格形函数$\Psi_I(\pmb{x})$的一阶和二阶导数分别为:
\begin{equation}
\begin{split}
    \Psi_{I,i}(x)=\left[\begin{matrix}
    p_{,i}^{[p]T}(x_I-x)A^{-1}(x)\phi_s(x_I-x)\\
    +p^{[p]T}(x_I-x)A_{,i}^{-1}\phi_s(x_I-x)\\
    +p^{[p]T}(x_I-x)A^{-1}(x)\phi _{s,i}(x_I-x)\\
    \end{matrix}\right]
    p^{[p]}(0)
\end{split}
\end{equation}
\begin{equation}
\begin{split}
    \Psi_{I,ij}(x)=\left[\begin{matrix}
    p_{,ij}^{[p]T}(x_I-x)A^{-1}(x)\phi_s(x_I-x)\\
    +p_{,i}^{[p]T}(x_I-x)A_{,j}^{-1}(x)\phi_s(x_I-x)\\
    +p_{,i}^{[p]T}(x_I-x)A^{-1}(x)\phi_{s,j}(x_I-x)\\
    +p^{[p]T}(x_I-x)A_{,ij}^{-1}(x)\phi_s(x_I-x)\\
    +p_{,j}^{[p]T}(x_I-x)A_{,i}^{-1}(x)\phi_s(x_I-x)\\
    +p^{[p]T}(x_I-x)A_{,i}^{-1}(x)\phi_{s,j}(x_I-x)\\
    +p^{[p]T}(x_I-x)A^{-1}(x)\phi_{s,ij}(x_I-x)\\
    +p_{,j}^{[p]T}(x_I-x)A^{-1}(x)\phi_{s,i}(x_I-x)\\
    +p^{[p]T}(x_I-x)A_{,j}^{-1}(x)\phi_{s,i}(x_I-x)\\
    \end{matrix}\right]
    p^{[p]}(0)
\end{split}
\end{equation}
式中$A_{,i}^{-1}=-A^{-1}A_{,i}A^{-1},A_{,ij}^{-1}=-A^{-1}(A_{,ij}A^{-1}+A_{,i}A_{,j}^{-1}+A_{,j}A_{,i}^{-1})$,可以看出无网格形函数及其导数的计算都较为复杂。\par
图 一维无网格形函数及其导数\\\par
图 二维无网格形函数及其导数\par
图。图。分别表示一维和二维情况下的无网格形函数及其导数图,从图中可以看出,无网格形函数在全域上连续光滑,但在无网格节点处,形函数不具有插值性,因此无法像有限元法一样直接施加本质边界条件。







\end{document}


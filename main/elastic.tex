\chapter{弹性力学问题HR弱形式}
\section{Hellinger-Reissner变分原理}
不失为一般性,在求解域$\Omega$内考虑如下弹性力学控制方程:
\begin{equation}\label{E control equation}
\begin{split}
\begin{cases}
    \sigma_{ij,j}+b_i=0&\text{in}\Omega\\
    \sigma_{ij}n_j=t_i&\text{on}\Gamma^t\\
    u_i=g_i&\text{on}\Gamma^g
\end{cases}
\end{split}
\end{equation}
其中,$u_i$和$\sigma_{ij}$为位移和应力分量,$b_i$为求解域$\Omega$上的体力分量。$\Gamma^g$和$\Gamma^t$分别为本质边界条件和自然边界条件,
并且$\Gamma^g\cup\Gamma^t=\partial\Omega,\Gamma^t\cap\Gamma^g=\varnothing,\partial\Omega$为求解域$\Omega$的边界。$t_i$和$g_i$分别为两类边界上已知的外力和位移分量,$n_i$表示为所在边界的外法向量分量。\par
对于式(\ref{E control equation})定义的弹性力学问题,存在以下余能泛函:
\begin{equation}
\begin{split}
    \pmb{\Pi}_c=\int_{\Omega}\pmb{B}(\sigma_{ij})d\Omega-\int_{\Gamma^g}\sigma_{ij}n_jg_id\Gamma
\end{split}
\end{equation}
其中$\pmb{B}(\sigma_{ij})$为应变余能,其与应力和应变之间的关系式为:
\begin{equation}
\begin{split}
\frac{\partial\pmb{B}(\sigma_{ij})}{\partial\sigma_{ij}}&=\varepsilon_{ij}\\
&=C^{-1}_{ijkl}\sigma_{kl}
\end{split}
\end{equation}
式中$C_{ijkl}$为四阶弹性张量\par
此时用特定的拉格朗日乘子$\lambda_i$和$\eta_i$,建立新的变分泛函
\begin{equation}\label{functional1}
\begin{split}
    \pmb{\Pi}_c^*&=\int_{\Omega}\pmb{B}(\sigma_{ij})d\Omega-\int_{\Gamma^g}\sigma_{ij}n_jg_id\Gamma\\
    &+\int_{\Omega}\lambda_i(\sigma_{ij,j}+b_i)d\Omega+\int_{\Gamma^t}\eta_i(\sigma_{ij}n_j-t_i)d\Gamma
\end{split}
\end{equation}
将式(\ref{functional1})进行一阶变分,同时把$\sigma_{ij}$和$\lambda_i,\eta_i$看作为独立变量,可以得到:
\begin{equation}\label{First-order variational}
\begin{split}
    \delta\pmb{\Pi}_c^*&=\int_{\Omega}(\frac{\partial\pmb{B}}{\partial\sigma_{ij}}\delta\sigma_{ij}+\delta\lambda_i(\sigma_{ij,j}+b_i)+\lambda_i\delta\sigma_{ij,j})d\Omega\\
&-\int_{\Gamma^g}\delta\sigma_{ij}n_jg_id\Gamma+\int_{\Gamma^t}\delta\eta_i(\sigma_{ij}-t_i)d\Gamma+\int_{\Gamma^t}\eta_i\delta\sigma_{ij}n_jd\Gamma
\end{split}
\end{equation}
通过格林公式和$\sigma_{ij}$的对称性,存在如下公式:
\begin{equation}\label{Green formula}
\begin{split}
    \int_{\Omega}\lambda_i\delta\sigma_{ij,j}d\Omega&=\int_{\Gamma}\lambda_i\delta\sigma_{ij}n_jd\Gamma-\int_{\Omega}\lambda_{i,j}\delta\sigma_{ij}d\Omega\\
&=\int_{\Gamma^g}\delta\sigma_{ij}n_j\delta_id\Omega+\int_{\Gamma^t}\delta\sigma_{ij}n_j\delta_id\Omega-\int_{\Omega}\frac{1}{2}(\lambda_{i,j}+\lambda_{j,i})\delta\sigma_{ij}d\Omega
\end{split}
\end{equation}
将式(\ref{Green formula})代入式(\ref{First-order variational})从而得到:
\begin{equation}
\begin{split}
    \delta\pmb{\Pi}_c^*&=\int_{\Omega}((\frac{\partial\pmb{B}}{\partial\sigma_{ij}}-\frac{1}{2}(\lambda_{i,j}+\lambda_{j,i}))\delta\sigma_{ij}+(\sigma_{ij,j}+b_i)\delta\lambda_i)d\Omega\\
    &+\int_{\Gamma^g}(\lambda_i-g_i)\delta\sigma_{ij}n_jd\Gamma+\int_{\Gamma^t}((\sigma_{ij}n_j-t_i)\delta\eta_i+(\lambda_i+\eta_i)\delta\sigma_{ij}n_j)d\Gamma
\end{split}
\end{equation}\par
根据泛函驻值条件$\delta\Pi_c^*=0$得出:
\begin{equation}
\begin{split}
    &\lambda_i=g_i=u_i\\
    &\eta_i=-u_i
\end{split}
\end{equation}\par
将$\lambda_i=u_i,\eta_i=-u_i$代入式(\ref{functional1})中得到根据存在双变量$(u_i,\sigma_{ij})$的Hellinger-Reissner变分原理,从而强形式(\ref{E control equation})所对应的能量泛函为:
\begin{equation}\label{energy functional1}
\begin{split}
    \pmb{\Pi}_{H\!R}(\sigma_{ij},u_i)&=\int_{\Omega}\pmb{B}(\sigma_{ij})d\Omega-\int_{\Gamma^g}\sigma_{ij}n_jg_id\Gamma\\
    &+\int_{\Omega}u_i(\sigma_{ij,j}+b_i)d\Omega-\int_{\Gamma^t}u_i(\sigma_{ij}n_j-t_i)d\Gamma
\end{split}
\end{equation}\par
对式(\ref{energy functional1})进行变分得到相对应的弱形式:
\begin{equation}\label{weak form1}
\begin{split} 
    \delta\pmb{\Pi}_{H\!R}(\sigma_{ij},u_i)&=\int_{\Omega}\delta\sigma_{ij}\frac{\partial\pmb{B}}{\partial \sigma_{ij}}d\Omega+\int_{\Omega}\delta\sigma_{ij,j}u_id\Omega-\int_{\Gamma^t}\delta\sigma_{ij}n_ju_id\Gamma\\
    &-\int_{\Gamma^g}\delta\sigma_{ij,j}n_jg_id\Gamma+\int_{\Omega}\delta u_i\sigma_{ij,j}d\Omega- \int_{\Gamma^t}\delta u_i\sigma_{ij}n_jd\Gamma\\
    &+\int_{\Omega}\delta u_ib_id\Omega+\int_{\Gamma^t}\delta u_it_id\Gamma=0
\end{split}
\end{equation}
对能量泛函$\pmb{\Pi}_{H\!R}$取极值时,要求对于任意的$\delta u_i$、$\delta\sigma_{ij}$关于$\delta\pmb{\Pi}_{H\!R}$都要恒成立,此时将弱形式(\ref{weak form1})根据$\delta u_i$、$\delta\sigma_{ij}$改写为下列两式:
\begin{equation}
\begin{split}
    \int_{\Gamma}\delta u_i\sigma_{ij}n_jd\Gamma&-\int_{\Omega}\delta u_i\sigma_{ij,j}d\Omega-\int_{\Gamma^g}\delta u_i\sigma_{ij}n_jd\Gamma
    =\int_{\Gamma^t}\delta u_it_id\Gamma+\int_{\Omega}\delta u_ib_id\Omega
\end{split}
\end{equation} 
\begin{equation}
\begin{split}
    \int_{\Omega}\delta\sigma_{ij}C^{-1}_{ijkl}\sigma_{kl}d\Omega=\int_{\Gamma}\delta\sigma_{ij}n_ju_id\Gamma-\int_{\Omega}\delta\sigma_{ij,j}u_id\Omega
    -\int_{\Gamma^g}\delta\sigma_{ij}n_ju_id\Gamma+\int_{\Gamma^g}\delta\sigma_{ij}n_jg_id\Gamma
\end{split}
\end{equation}    
\section{位移离散和应力离散}
\subsection{位移离散与再生核近似}
位移分量$u_i$采用基于再生核近似的无网格形函数进行离散。无网格法通过如图所示的问题域$\Omega$和边界$\Gamma$上布置一系列无网格节点$\{\pmb{x}_I\}^{N\!P}_{I=1}$进行离散,
其中$N\!P$表示无网格节点数量。每个无网格节点$\pmb{x}_I$对应的形函数为$\Psi(\pmb{x})$,影响域为$supp(\pmb{x}_I)$,
每一个节点的影响域$supp(\pmb{x}_I)$满足$\Omega\in^{N\!P}_{I=1}supp(\pmb{x}_I)$。不失为一般性,考虑任意位移分量$u_i$,其对应的无网格近似函数$u^h_i$表示为:
\begin{equation}
\begin{split}
    u^h_i(\pmb{x})=\sum_{I=1}^{N\!P}\Psi_I(\pmb{x})d_{iI}
\end{split}
\end{equation}
其中,$d_{iI}$表示与无网格节点$\pmb{x}_I$对应的系数\par
根据再生核近似理论[],无网格形函数可以假设为:
\begin{equation}\label{shapefunction}
\begin{split}
    \Psi_I(\pmb{x})=\sum_{I=1}^{N\!P}\pmb{p}^T(\pmb{x}_I-\pmb{x})\pmb{c}(\pmb{x})\phi_s(\pmb{x}_I-\pmb{x})
\end{split}
\end{equation}
式中,$\pmb{p}(\pmb{x})$表示为$p$阶的多项式基函数向量,表达式为:
\begin{equation}
\begin{split}
    \pmb{p}(\pmb{x})=\{1,x,y,\dotsb,x^iy^i,\dotsb,y^p\}.0\le i+j \le p
\end{split}
\end{equation}
而$\phi_s(\pmb{x}_I-\pmb{x})$是附属于节点$\pmb{x}_I$的核函数,其影响域的大小由影响域尺寸$s$决定,核函数以及其影响域的大小共同决定了无网格形函数的局部紧支性和光滑性。对应二维问题,一般情况下核函数$\phi_s(\pmb{x}_I-\pmb{x})$的影响域为圆形域或者矩形域,可由下列公式得到:
\begin{equation}
\begin{split}
    \phi_s(\pmb{x}_I-\pmb{x})=\phi_{s_x}(r_x)\phi_{s_y}(r_y),r_x=\frac{\lvert x_I-x\rvert}{s_x},r_y=\frac{\lvert y_I-y \rvert}{s_y}
\end{split}
\end{equation}
其中$s_x$和$s_y$分别为$x$和$y$方向上影响域的大小,计算时一般使得两个方向上的影响域大小相等即$s_x=s_y=s$。选取核函数时一般遵循核函数阶次$m$大于等于基函数阶次$p(m\ge p)$的原则。针对二阶势问题的弹性力学问题,无网格基函数一般选择二阶或者三阶,而核函数$\phi_s(\pmb{x}_I-\pmb{x})$则选取三次样条函数:
\begin{equation}
\begin{split}
    \phi(r)=\frac{1}{3!}
\begin{cases}
    (2-2r)^3-4(1-2r)^3 &r\le \frac{1}{2}\\
    (2-2r)^3&\frac{1}{2}<r\le 1\\
    0&r>1
\end{cases}
\end{split}
\end{equation}
无网格形函数表达式(\ref{shapefunction})中的$\pmb{c}$为待定系数向量,该表达式可以通过满足再生条件确定:
\begin{equation}\label{regeneration conditions}
\begin{split}
    \sum_{I=1}^{N\!P}\Psi_I(\pmb{x})\pmb{p}(\pmb{x}_I-\pmb{x})=\pmb{0}
\end{split}
\end{equation}
将无网格形函数表达式(\ref{shapefunction})代入再生条件(\ref{regeneration conditions})中,可以得到:
\begin{equation}
\begin{split}
    \pmb{c}(\pmb{x})=\pmb{A}^{-1}(\pmb{x})\pmb{p}(\pmb{0})
\end{split}
\end{equation}
其中$\pmb{A}(\pmb{x})$表示矩量矩阵,表达式为:
\begin{equation}
\begin{split}
    \pmb{A}(\pmb{x})=\sum_{I=1}^{N\!P}\pmb{p}(\pmb{x}_I-\pmb{x})\pmb{p}^T(\pmb{x}_I-\pmb{x})\phi_s(\pmb{x}_I-\pmb{x})
\end{split}
\end{equation}\par
将$\pmb{c}(\pmb{x})$代入到式(\ref{shapefunction})中得到再生核无网格形函数的表达式:
\begin{equation}
\begin{split}
    \Psi_I(\pmb{x})=\pmb{p}^T(\pmb{0})\pmb{A}^{-1}(\pmb{x})\pmb{x}_I-\pmb{x}\phi_s(\pmb{x}_I-\pmb{x})
\end{split}
\end{equation}\par
无网格形函数$\Psi_I(\pmb{x})$的一阶和二阶导数分别为:
\begin{equation}
\begin{split}
    \Psi_{I,i}(x)=\left[\begin{matrix}
    p_{,i}^{[p]T}(x_I-x)A^{-1}(x)\phi_s(x_I-x)\\
    +p^{[p]T}(x_I-x)A_{,i}^{-1}\phi_s(x_I-x)\\
    +p^{[p]T}(x_I-x)A^{-1}(x)\phi _{s,i}(x_I-x)\\
    \end{matrix}\right]
    p^{[p]}(0)
\end{split}
\end{equation}
\begin{equation}
\begin{split}
    \Psi_{I,ij}(x)=\left[\begin{matrix}
    p_{,ij}^{[p]T}(x_I-x)A^{-1}(x)\phi_s(x_I-x)\\
    +p_{,i}^{[p]T}(x_I-x)A_{,j}^{-1}(x)\phi_s(x_I-x)\\
    +p_{,i}^{[p]T}(x_I-x)A^{-1}(x)\phi_{s,j}(x_I-x)\\
    +p^{[p]T}(x_I-x)A_{,ij}^{-1}(x)\phi_s(x_I-x)\\
    +p_{,j}^{[p]T}(x_I-x)A_{,i}^{-1}(x)\phi_s(x_I-x)\\
    +p^{[p]T}(x_I-x)A_{,i}^{-1}(x)\phi_{s,j}(x_I-x)\\
    +p^{[p]T}(x_I-x)A^{-1}(x)\phi_{s,ij}(x_I-x)\\
    +p_{,j}^{[p]T}(x_I-x)A^{-1}(x)\phi_{s,i}(x_I-x)\\
    +p^{[p]T}(x_I-x)A_{,j}^{-1}(x)\phi_{s,i}(x_I-x)\\
    \end{matrix}\right]
    p^{[p]}(0)
\end{split}
\end{equation}
式中$A_{,i}^{-1}=-A^{-1}A_{,i}A^{-1},A_{,ij}^{-1}=-A^{-1}(A_{,ij}A^{-1}+A_{,i}A_{,j}^{-1}+A_{,j}A_{,i}^{-1})$,可以看出无网格形函数及其导数的计算都较为复杂。\par
图 一维无网格形函数及其导数\\\par
图 二维无网格形函数及其导数\par
图。图。分别表示一维和二维情况下的无网格形函数及其导数图,从图中可以看出,无网格形函数在全域上连续光滑,但在无网格节点处,形函数不具有插值性,因此无法像有限元法一样直接施加本质边界条件。
\subsection{应力离散与再生光滑梯度近似}
应力分量$\sigma_{ij}$采用在每个背景积分单元内建立局部的多项式进行离散。考虑如图所示二维三角形背景积分单元,将求解域$\Omega$划分为一系列背景积分单元$\Omega_C$,$C=1,2,\dotsb,n_c$,并且$\cup_{C=1}^{n_c}\Omega_C=\Omega$。
在背景积分单元$\omega_C$内,假设应力分量$\sigma_{ij}$为任意的$p$阶多项式
% \chapter{绪论}
\chapter{薄板问题}
\section{薄板控制方程}
考虑厚度为$h$的薄板,根据Kirchhoff薄板假设将薄板$x,y,z$方向上的总位移定义为$\hat{\pmb{u}}(\pmb{x})$:
此时,位移$\hat{\pmb{u}}(\pmb{x})$可以表示为:
\begin{equation}
    \begin{split}
    \begin{cases}
        \hat{u}_{\alpha}(\pmb{x})=u_{\alpha}(x_1,x_2)-x_3w_{,\alpha}\quad \alpha=1,2\\
        \hat{u}_3(\pmb{x})=w(x_1,x_2)
    \end{cases}
    \end{split}
    \end{equation}
式中$u_{\alpha}$表示薄板中面处$x,y$方向上的位移,$w$表示挠度。\par
根据小变形假设关系式:
\begin{equation}\label{line elasticity}
\begin{split}
    \hat{\varepsilon}_{\alpha\beta}=\frac{1}{2}(\hat{u}_{\alpha,\beta}+\hat{u}_{\beta,\alpha})
\end{split}
\end{equation}
其中,${\varepsilon}$为应变。\par
根据式(\ref{line elasticity})得出有关Kirchhoff薄板假设的应变关系式如下:
\begin{equation}\label{strain}
\begin{split}
\begin{cases}
    \hat{\varepsilon}_{\alpha\beta}=\frac{1}{2}(\hat{u}_{\alpha,\beta}+\hat{u}_{\beta,\alpha})=\varepsilon_{\alpha\beta}+\kappa_{\alpha\beta}x_3 \quad \alpha,\beta=1,2\\
    \hat{\varepsilon}_{3i}=\hat{\varepsilon}_{i3}=0 \quad i=1,2,3
\end{cases}
\end{split}
\end{equation}
其中:
\begin{equation}
\begin{split}
    \varepsilon_{\alpha\beta}=\frac{1}{2}(u_{\alpha,\beta}+u_{\beta,\alpha}),\kappa_{\alpha\beta}=-w_{,\alpha\beta}
\end{split}
\end{equation}
式中,$\kappa_{\alpha\beta}=-w_{,\alpha\beta}$为曲率张量$\pmb{\kappa}$的分量\par
考虑经典的线弹性本构关系:
\begin{equation}\label{stress}
\begin{split}
    &\pmb{\sigma}=\pmb{C}\pmb{:}\pmb{\varepsilon}\\
    &\hat{\sigma}_{\alpha\beta}=C_{\alpha\beta\gamma\eta}\hat{\varepsilon}_{\alpha\beta}=C_{\alpha\beta\gamma\eta}(\varepsilon_{\gamma\eta+x_3\kappa_{\gamma\eta}})
\end{split}
\end{equation}
其中$\hat{\sigma}_{\alpha\beta}$为应力,$C_{\alpha\beta\gamma\eta}$为四阶弹性张量\par
根据最小势能原理得出的势能泛函关系为:
\begin{equation}\label{potential energy}
\begin{split}
    \pmb{\Pi}(\hat{\pmb{u}})=\frac{1}{2}\int_{\hat{\Omega}}\hat{\varepsilon}_{\alpha\beta}C_{\alpha\beta\gamma\eta}\hat{\varepsilon}_{\gamma\eta}d\hat{\Omega}-\int_{\hat{\Omega}}\hat{\pmb{u}}\cdot\pmb{b}d\hat{\Omega}-\int_{\Gamma^t}\hat{\pmb{u}}\cdot\pmb{t}d\Gamma
\end{split}
\end{equation}
将式(\ref{stress})、(\ref{strain})代入式(\ref{potential energy})中的第一项可以得到:
\begin{equation}\label{energy relations}
\begin{split}
&\int_{\hat{\Omega}}\frac{1}{2}\hat{\varepsilon}_{\alpha\beta}\hat{\sigma}_{\alpha\beta}d\Omega=\int_{\Omega}\int_{\frac{-h}{2}}^{\frac{h}{2}}\frac{1}{2}(\varepsilon_{\alpha\beta}+\kappa_{\alpha\beta}x_3)C_{\alpha\beta\gamma\eta}(\varepsilon_{\gamma\eta}+x_3\kappa_{\gamma\eta})dx_3d\Omega\\
&=\int_{\Omega}\frac{1}{2}h\varepsilon_{\alpha\beta}C_{\alpha\beta\gamma\eta}\varepsilon_{\gamma\eta}d\Omega+\int_{\Omega}\frac{1}{2}\kappa_{\alpha\beta}\frac{h^3}{12}C_{\alpha\beta\gamma\eta}\kappa_{\gamma\eta}d\Omega\\
&=\int_{\Omega}\frac{1}{2}h\varepsilon_{\alpha\beta}\sigma_{\alpha\beta}d\Omega+\int_{\Omega}\frac{1}{2}\kappa_{\alpha\beta}M_{\alpha\beta}d\Omega
\end{split}
\end{equation}\par
根据式(\ref{energy relations})可以划分为两个独立的问题。即具有变量$\hat{u}_{\alpha}$的传统弹性力学问题和具有变量$w$的薄板问题,能量泛函关系式(\ref{potential energy})拆分为:
\begin{equation}
\begin{split}
\Pi(\hat{\pmb{u}})=\Pi_E(\hat{\pmb{u}})+\Pi_P(M_{\alpha\beta},w)
\end{split}
\end{equation}
\section{弹性力学问题伽辽金弱形式}
传统弹性力学问题的势能泛函的表达式为:
\begin{equation}\label{elasticity}
\begin{split}
    \Pi_E(\hat{\pmb{u}})=\int_{\Omega}\frac{1}{2}\varepsilon_{\alpha\beta}\sigma_{\alpha\beta}d\Omega-\int_{\Omega}\hat{\pmb{u}}\cdot\pmb{b}d\Omega-\int_{\Gamma^t}\hat{\pmb{u}}\cdot\pmb{t}d\Gamma
\end{split}
\end{equation}
这里采用拉格朗日乘子法在伽辽金无网格法施加强制边界条件,即在弹性力学问题的势能泛函(\ref{elasticity})中引入位移强制边界条件对应的约束项,相应的势能泛函为:
\begin{equation}\label{Elambda}
\begin{split}
    \bar{\Pi}_E(\hat{\pmb{u}},\pmb{\lambda})=\Pi^E(\hat{\pmb{u}})-\int_{\Gamma^g}\pmb{\lambda}(\hat{\pmb{u}}-\pmb{g})d\Gamma
\end{split}
\end{equation}
其中$\pmb{\lambda}=\{\lambda_1,\dotsb,\lambda_{n_{sd}}\}^T$为拉格朗日乘子,$\Pi^E(\hat{\pmb{u}})$是式(\ref{elasticity})定义的泛函。式(\ref{Elambda})所表示的泛函的驻值条件为:
\begin{equation} 
\begin{split}
    \delta\bar{\Pi}^E(\hat{\pmb{u}},\lambda)&=\delta\Pi^E(\hat{\pmb{u}})-\int_{\Gamma^u}\delta\hat{\pmb{u}}~\lambda d\Gamma-\int_{\Gamma^g}\delta\lambda~(\hat{\pmb{u}}-\pmb{g})\\
    &=\int_{\Omega}\frac{1}{2}\varepsilon_{\alpha\beta}\sigma_{\alpha\beta}d\hat{\Omega}-\int_{\Omega}\hat{\pmb{u}}~\pmb{b}d\hat{\Omega}-\int_{\Gamma^t}\hat{\pmb{u}}~\pmb{t}d\Gamma\\
    &-\int_{\Gamma^g}\delta\hat{\pmb{u}}\lambda d\Gamma-\int_{\Gamma^g}\delta\lambda~(\hat{\pmb{u}}-\pmb{g})d\Gamma\\
    &=0
\end{split}
\end{equation}
\section{薄板问题伽辽金弱形式}
薄板问题控制方程:
\begin{equation}
\begin{split}
\begin{cases}
    M_{\alpha\beta,\alpha\beta}+\bar q=0  &\text{in}~\Omega\\
    w=\bar{w} &\text{on}~\Gamma^w\\
    w_{,\pmb{n}}=\bar \theta_{\pmb{n}}&\text{on}~\Gamma^{\theta}\\
    M_{\pmb{nn}}=\bar M_{\pmb{nn}}&\text{on}~\Gamma^M\\
    Q_{\pmb{n}}+M_{\pmb{ns,s}}=\bar V_{\pmb{n}}&\text{on}~\Gamma^V\\
    w_i=\bar{w}_i&i=1,2,3\cdots\\
    P_j=\bar P_j&j=1,2,3\cdots
\end{cases}
\end{split}
\end{equation}

\begin{equation}
\begin{split}
    &w_{,\pmb{n}}=w_{,\alpha}n_{\alpha}\\
    &M_{\pmb{nn}}=M_{\alpha\beta}n_{\alpha}n_{\beta}\\
    &Q_{\pmb{n}}=M_{\alpha\beta,\beta}n_{\alpha}\\
    &M_{\pmb{ns},\pmb{s}}=M_{\alpha\beta,\beta}n_{\alpha}\\
    &M_{\pmb{ns},\pmb{s}}=M_{\alpha\beta,\gamma}s_{\alpha}n_{\beta}s_{\gamma}\\
    &P_j=M_{\alpha\beta}(s^1_{\alpha}n^1_{\beta}-s^2_{\alpha}n^2_{\beta})
\end{split}
\end{equation}\par
薄板问题的势能泛函关系式:
\begin{equation}
\begin{split}
    \Pi_P(M_{\alpha\beta},w)&=\int_{\Omega}\frac{1}{2}M_{\alpha\beta}D^{-1}_{\alpha\beta\gamma\eta}M_{\gamma\eta}d\Omega+\int_{\Omega}wqd\Omega-\int_{\Gamma_w}\bar{V}_{\pmb{n}}wd\Gamma+\int_{\Gamma_{\theta}}\bar{M}_{\pmb{nn}}\theta_{\pmb{n}}d\Gamma
\end{split}
\end{equation}\par
引入拉格朗日乘子:
\begin{equation}
\begin{split}
    \Pi_P(M_{\alpha\beta},w,\lambda)&=\int_{\Omega}\frac{1}{2}M_{\alpha\beta}D^{-1}_{\alpha\beta\gamma\eta}M_{\gamma\eta}d\Omega+\int_{\Omega}wqd\Omega+\int_{\Gamma_w}\lambda^w(w-\bar{w})d\Gamma\\
    &+\int_{\Gamma_{\theta}}\lambda^{\theta}(\theta_{\pmb{n}}-\bar{\theta}_{\pmb{n}})d\Gamma-\int_{\Gamma_w}\bar{V}_{\pmb{n}}wd\Gamma+\int_{\Gamma_{\theta}}\bar{M}_{\pmb{nn}}\theta_{\pmb{n}}d\Gamma
\end{split}
\end{equation}\par
弱形式:
\begin{equation}
\begin{split}
    &\int_{\Omega}\delta M_{\alpha\beta}D^{-1}_{\alpha\beta\gamma\eta}M_{\gamma\eta}d\Omega+\int_{\Omega}\delta wqd\Omega+\int_{\Gamma_w}\delta\lambda^w(w-\bar{w})d\Gamma+\int_{\Gamma_w}\lambda^w\delta wd\Gamma\\
    &+\int_{\Gamma_{\theta}}\delta\lambda^{\theta}(\theta_{\pmb{n}}-\bar{\theta_{\pmb{n}}})d\Gamma+\int_{\Gamma_{\theta}}\lambda^{\theta}\delta\theta_{\pmb{n}}d\Gamma-\int_{\Gamma_w}\delta\bar{V}_{\pmb{n}}wd\Gamma+\int_{\Gamma_{\theta}}\delta\bar{M}_{\pmb{nn}}\theta_{\pmb{n}}d\Gamma\\
    &=0
\end{split}
\end{equation}
\documentclass[11pt,a4paper]{article}
\usepackage{ctex,amsmath,titlesec}
\usepackage[text={140mm,250mm},centering]{geometry}
\usepackage[margin=20mm]{geometry}
% \setlength{\abovedisplayskip}{3pt}
% \setlength{\belowdisplayskip}{3pt}
% \setlength{\lineskiplimit}{0pt}
% \setlength{\lineskip}{0pt}
% \setlength{\abovecaptionskip}{2pt plus 1pt minus 1pt}
% \setlength{\belowcaptionskip}{2pt plus 1pt minus 1pt}
% \setlength{\abovedisplayshortskip}{2pt plus 1pt minus 1pt}
% \setlength{\belowdisplayshortskip}{2pt plus 1pt minus 1pt}
\begin{document}
\title{Hellinger-Reissner的动力分析}
\date{}
\maketitle
\section{Hellinger-Reissner变分原理}
不失为一般性,在求解域$\Omega$内考虑如下弹性力学问题控制方程
\begin{displaymath}  
\begin{equation}
    \begin{split}
    \begin{cases}
        \sigma_{ij,j}+b_i=0&in\Omega\\
        \sigma_{ij}n_j=t_i&on\Gamma^t\\
        u_i=g_i&on\Gamma^g
    \end{cases}
\end{split}
\end{equation}
\end{displaymath}
其中,$u_i$和$\sigma_{ij}$为位移和应力分量,$b_i$为求解域$\Omega$上的体力分量。$\Gamma^g$和$\Gamma^t$分别为本质边界条件和自然边界条件。并且$\Gamma^g\cup\Gamma^t=\partial\Omega,\Gamma^t\cap\Gamma^g=\varnothing,\partial\Omega$
为求解域$\Omega$的边界。$t_i$和$g_i$分别为两类边界上已知的外力和位移分量,$n_i$表示为所在边界的外法向量分量。\par
根据Hellinger-Reissner原理,强形式(1)所对应的能量泛函为
\begin{displaymath}
\begin{equation}
    \begin{split}
\Pi_{H\!R}(\sigma_{ij},u_i)=\int_{\Omega}W(\sigma_{ij})d\Omega-\int_{\Gamma^g}\sigma_{ij}n_jg_id\Gamma+\\\int_{\Omega}u_i(\sigma_{ij,j}+b_i)d\Omega-\int_{\Gamma^t}u_i(\sigma_{ij}n_j-t_i)d\Gamma
    \end{split}
\end{equation}
\end{displaymath}
式子中$W$为弹性体的余能密度函数,其与应力之间的关系式为:
\begin{displaymath}
\begin{equation}
    \begin{split}
        \frac{\partial W}{\partial \sigma_{ij}}=C^{-1}_{ijkl}\sigma_{kl}
    \end{split}
\end{equation}
\end{displaymath}
其中$C_{ijkl}$为四阶弹性张量\par
对上述式子进行变分可以得到与之相对应的弱形式为:

\begin{displaymath}
\begin{equation}
\begin{split} 
    \delta \Pi_{H\!R}=&\int_{\Omega}\delta\sigma_{ij}\frac{\partial W}{\partial \sigma_{ij}}d\Omega+
    \int_{\Omega}\delta\sigma_{ij,j}u_id\Omega-\int_{\Gamma^t}\delta\sigma_{ij}n_ju_id\Gamma-\\
    &\int_{\Gamma^g}\delta\sigma_{ij,j}n_jg_id\Gamma+\int_{\Omega}\delta u_i\sigma_{ij,j}d\Omega- \int_{\Gamma^t}\delta u_i\sigma_{ij}n_jd\Gamma+\\
    &\int_{\Omega}\delta u_ib_id\Omega+\int_{\Gamma^t}\delta u_it_id\Gamma
    =0
\end{split}
\end{equation}
\end{displaymath}\\

 当对能量泛函$\Pi_{H\!R}$取极值,对于任意的$\delta u_i,\delta \sigma_{ij}$,式子(4)都恒成立。利用几何关系$\Gamma^t\cup\Gamma^g=\partial\Omega,\Gamma^t\cap\Gamma^g=\varnothing$,将式子(4)改写为:
\begin{displaymath}
\begin{equation}
\begin{split}
\int_{\Omega}\delta\sigma_{ij}C^{-1}_{ijkl}\sigma_{kl}d\Omega=\int_{\Gamma}\delta\sigma_{ij}n_ju_id\Gamma-\int_{\Omega}\delta\sigma_{ij,j}u_id\Omega\\
-\int_{\Gamma^g}\delta\sigma_{ij}n_ju_id\Gamma+\int_{\Gamma^g}\delta\sigma_{ij}n_jg_id\Gamma\\
\end{split}
\end{equation} 
\end{displaymath}

\begin{displaymath}
\begin{equation}
\begin{split}
 \int_{\Gamma}\delta u_i\sigma_{ij}n_jd\Gamma&-\int_{\Omega}\delta u_i\sigma_{ij,j}d\Omega-\int_{\Gamma^g}\delta u_i\sigma_{ij}n_jd\Gamma\\
 &=\int_{\Gamma^t}\delta u_it_id\Gamma+\int_{\Omega}\delta u_ib_id\Omega
 \end{split}
\end{equation} 
\end{displaymath}
\section{再生核无网格形函数}
无网格法通过在分析区域$\Omega$和边界$\Gamma$上布置一系列节点$\{\pmb x_I\}^{N\!P}_{I=1}$实现空间离散,其中NP代表无网格节点数量。每个无网格节点$\pmb x_I$对应的形函数为$\pmb \Psi_I(x)$,影响域为$supp(\pmb x_I)$。
考虑任意变量$u(\pmb x,t)$,其对应的无网格近似函数$u^h(\pmb x,t)$为:
\begin{displaymath}
    \begin{equation}
        \begin{split}
        u^h(\pmb x,t)=\sum_{I=1}^{NP}\pmb \Psi_I(x)d_I(t)    
        \end{split}
    \end{equation}      
\end{displaymath}
其中,$t$为时间;$d_I$表示与节点$\pmb x_I$对应的系数。\par
根据再生核近似理论,无网格形函数可以假设为如下形式:
\begin{dispalmath}
\begin{equation}
\begin{split}
    \pmb \Psi_I(\pmb x)=\pmb p^T(\pmb x_I-\pmb x)\pmb{c(x)}\phi_{sI}(\pmb x_I-x)
\end{split}
\end{equation}
其中,$\pmb {c(x)}$为待定系数向量;$\pmb {p(x)}$为$p$阶单项式基向量,并且
\begin{dispalymath}
\begin{equation}
\begin{split}
    \pmb {p(x)}=\{1,x,y,\dotsb,x^iy^j,\dotsb,y^p\},0\le i+j \le p
\end{split}
\end{equation}
\end{displaymath}
其中$\psi_{sI}(\pmb x_I-\pmb x)$是附属于节点$x_I$的核函数,其影响域的大小由影响域的尺寸$s_I$确定,核函数及其影响域
的大小共同决定了无网格形函数的局部紧支性和光滑性。本研究分别采用三次和五次B样条核函数进行计算,二维问题的影响域为张量
积形式矩形域。\par
将一致型条件引入式子(8),即可得到位置系数$c(\pmb x)$,进一步得到无网格形函数表达式:
\begin{displaymath}
\begin{equation}
\begin{split}
    \pmb \Psi_I(\pmb x)=\pmb p^T(0)\pmb A^{-1}(\pmb x)(\pmb x_I-\pmb x)\psi(\pmb x_I-\pmb x)
\end{split}
\end{equation}
\end{displaymath}
式子中$\pmb {A(x)}$为矩量矩阵,并且
\begin{displaymath}
\begin{equation}
\begin{split}
        \pmb {A(x)}=\sum_{I=1}^{NP}\pmb p(\pmb x_I-\pmb x)\pmb p^T(\pmb x_I-\pmb x)\psi(\pmb x_I-\pmb x)
\end{split}
\end{equation}
\end{displaymath}
对式(4)进行微分,可以得到无网格形函数的一阶梯度和二阶梯度表达式
\begin{displaymath}
\begin{equation}
\begin{split}
    \Psi_{I,i}(x)=\left[\begin{matrix}
        p_{,i}^{[p]T}(x_I-x)A^{-1}(x)\phi(x_I-x)\\
        +p^{[p]T}(x_I-x)A_{,i}^{-1}\phi(x_I-x)\\
        +p^{[p]T}(x_I-x)A^{-1}(x)\phi _{,i}(x_I-x)\\
        \end{matrix}\right]
        p^{[p]}(0)
\end{split}
\end{equation}
\end{displaymath}
\begin{displaymath}
    \begin{equation}
    \begin{split}
        \Psi_{I,ij}(x)=\left[\begin{matrix}
            p_{,ij}^{[p]T}(x_I-x)A^{-1}(x)\phi(x_I-x)\\
            +p_{,i}^{[p]T}(x_I-x)A_{,j}^{-1}(x)\phi(x_I-x)\\
            +p_{,i}^{[p]T}(x_I-x)A^{-1}(x)\phi_{,j}(x_I-x)\\
            +p^{[p]T}(x_I-x)A_{,ij}^{-1}(x)\phi(x_I-x)\\
            +p_{,j}^{[p]T}(x_I-x)A_{,i}^{-1}(x)\phi(x_I-x)\\
            +p^{[p]T}(x_I-x)A_{,i}^{-1}(x)\phi_{,j}(x_I-x)\\
            +p^{[p]T}(x_I-x)A^{-1}(x)\phi_{,ij}(x_I-x)\\
            +p_{,j}^{[p]T}(x_I-x)A^{-1}(x)\phi_{,i}(x_I-x)\\
            +p^{[p]T}(x_I-x)A_{,j}^{-1}(x)\phi_{,i}(x_I-x)\\
            \end{matrix}\right]
            p^{[p]}(0)
    \end{split}
    \end{equation}
    \end{displaymath}\par
根据上述表达式可以看出,无网格形函数一般为有理式,因此其梯度计算复杂耗时。
\section{再生光滑梯度理论}
假设场变量$u(\pmb x,t)$为任意的$p$阶多项式,则其梯度$u_{,i}(\pmb x,t)$可以表示为:
\begin{displaymath}
\begin{equation}
\begin{split}
    u_{,i}(\pmb x,t)=\pmb a_i^T(t)\pmb q(\pmb x)
\end{split}
\end{equation}
\end{displaymath}
其中:$\pmb a_i$表示为任意系数向量;$\pmb q(\pmb x)$为$(p-1)$阶的单项式基向量,即$\pmb q(\pmb x)=\pmb p^{[p-1]}(\pmb x)$。根据文献可以知道$(p-1)$阶积分约束条件为:
\begin{displaymath}
\begin{equation}
\begin{split}
    \int_{\Omega}\pmb \Psi_{I,i}\pmb qd\Omega=\int_{\Gamma}\pmb \Psi_I \pmb qn_id\Gamma-\int_{\Omega}\pmb \Psi_I\pmb q_{,i}d\Omega
\end{split}
\end{equation}
\end{displaymath}
根据式子可以看出,由于$p$次基函数,伽辽金弱形式所采用的数值积分方法只有满足$(p-1)$阶积分约束条件,无网格数值解才能重现对应的多项式精确解。\par
根据再生光滑梯度理论,与再生核无网格形函数类似,无网格形函数$\pmb \Psi_I$的再生光滑梯度$\tilde{\pmb\Psi}_{I,i}$可表示为如下形式:
\begin{displaymath}
\begin{equation}
\begin{split}
\tilde{\pmb \Psi}_{I,i}(\pmb x)=\pmb q^T(\pmb x)\pmb c_i(\pmb x_I)\tilde{\varphi}(x)
\end{split}
\end{equation}
\end{displaymath}
其中:$\pmb c_i$为待定系数向量;$\tilde{\varphi}(\pmb x)$为核函数,这里取为
\begin{displaymath}
\begin{equation}
\begin{split}
    \tilde{\varphi}(\pmb x)=\begin{cases}
        1,&\pmb x\in\Omega_c\\
        0,&\pmb x\notin\Omega_c\end{cases}
\end{split}
\end{equation}
\end{displaymath}
其中:$\Omega_C$为互相不重叠且$\cup^{N_C}_{C=1}\Omega_C=\Omega$的积分单元;
$N_C$表示的是积分单元的总个数。图1给出了再生光滑梯度无网格法采用采用的三角形背景积分单元。不失为一般性,这里以三次基函数为例详细阐明
再生光滑梯度构造过程。当采用三次基函数$\pmb p(\pmb x)$时,$\pmb q(\pmb x)$的表达式为:
\begin{displaymath}
\begin{equation}
\begin{split}
    \pmb q(\pmb x)=(1,x,y,x^2,y^2,xy)^T\quad\pmb x\in\Omega_C
\end{split}
\end{equation}
\end{displaymath}
将式(18)代入到积分约束条件式(15)中,可得到:
\begin{displaymath}
\begin{equation}
\begin{split}
    \int_{\Omega_C}\pmb \Psi_{I,i}\pmb qd\Omega=\tilde{\pmb g}^C_{iI},I=1,2,\dotsb,N\!P
\end{split}
\end{equation}
\end{displaymath}
\begin{displaymath}
    \begin{equation}
    \begin{split}
     \tilde{\pmb g}^C_{iI}=\int_{\Gamma_C}\pmb \Psi_I\pmb qn_id\Gamma-\int_{\Omega_C}\pmb \Psi_I\pmb q_{,i}d\Omega
    \end{split}
    \end{equation}
    \end{displaymath}
再用式子(16)定义的光滑梯度$\tilde{\pmb \Psi}_{I,i}$替换到式子(19)中的标准梯度$\pmb \Psi_{I,i}$,可以得到$\pmb c_i=\pmb G^{-1}_C\:\tilde{\pmb g}^C_{iI}$
其中$\pmb G_C$为再生光滑梯度的矩量矩阵,并且
\begin{displaymath}
\begin{equation}
\begin{split}
    \pmb G_C=\int_{\Omega_C}\pmb q\pmb q^Td\Omega
\end{split}
\end{equation}
\end{displaymath}











\end{document}